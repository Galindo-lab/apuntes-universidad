% Created 2022-09-14 mié 12:03
% Intended LaTeX compiler: pdflatex
\documentclass[12pt]{article}
\usepackage[utf8]{inputenc}
\usepackage[T1]{fontenc}
\usepackage{graphicx}
\usepackage{grffile}
\usepackage{longtable}
\usepackage{wrapfig}
\usepackage{rotating}
\usepackage[normalem]{ulem}
\usepackage{amsmath}
\usepackage{textcomp}
\usepackage{amssymb}
\usepackage{capt-of}
\usepackage{hyperref}
\usepackage[spanish]{babel}
\usepackage{graphicx,geometry}
\geometry{ a4paper, left=1in, right=1in, top=1in, bottom=1in }
\renewcommand\familydefault{\sfdefault}
\usepackage{sectsty}
\sectionfont{\normalfont\Large }
\subsectionfont{\normalfont}
\usepackage{tabularx}
\usepackage{listings}
\lstdefinestyle{mystyle}{
numbers=left,
showspaces=false,
frame=leftline,
showspaces=false,
showstringspaces=false,
showtabs=false,
numberstyle=\tiny,
}
\lstset{
style=mystyle,
literate={á}{{\'a}}1
{é}{{\'e}}1
{í}{{\'{\i}}}1
{ó}{{\'o}}1
{ú}{{\'u}}1
{Á}{{\'A}}1
{É}{{\'E}}1
{Í}{{\'I}}1
{Ó}{{\'O}}1
{Ú}{{\'U}}1
{ü}{{\"u}}1
{Ü}{{\"U}}1
{ñ}{{\~n}}1
{Ñ}{{\~N}}1
{¿}{{?``}}1
{¡}{{!``}}1
}
\makeatletter
\usepackage{fancyhdr}
\pagestyle{fancy}
\usepackage{mdframed}
\BeforeBeginEnvironment{minted}{\begin{mdframed}}
\AfterEndEnvironment{minted}{\end{mdframed}}
\author{Luis Eduardo Galindo Amaya (1274895)}
\date{2022-09-13}
\title{Aritmética Binaria}
\hypersetup{
 pdfauthor={Luis Eduardo Galindo Amaya (1274895)},
 pdftitle={Aritmética Binaria},
 pdfkeywords={},
 pdfsubject={},
 pdfcreator={Emacs 26.3 (Org mode 9.1.9)}, 
 pdflang={Spanish}}
\begin{document}



\newcommand{\docente}{Arturo Arreola Alvarez}
\newcommand{\asignatura}{Organización de Computadoras (331)}
\newcommand{\semestre}{2022-2}

\newcommand{\miportada}[1]{
	\begin{titlepage}
		\vspace*{0.75in}
		\begin{flushleft}
			\sffamily
			\large #1       \\
			\Huge 
            \@title         \\
			\hrulefill
			\vspace{0.25in} \\
			\Large \@author \\
			\vspace*{\fill}
            \includegraphics[width=\textwidth]{../includes/filler.png} \\
			\vspace*{\fill}
			\large
			\begin{tabular}{|l|l|}
              \hline
			  Asignatura & \asignatura \\
			  Docente    & \docente    \\
			  Fecha      & \@date      \\
              \hline
			\end{tabular}
		\end{flushleft}
	\end{titlepage}
}

\miportada{ Práctica 4 }

\fancyhf{}
\lhead{ \asignatura }
\rhead{ \semestre }
\rfoot{Página \thepage}

\setlength\parindent{0pt}   % eliminar el intentado
\setlength{\parskip}{1.2em}
\maketitle

\section*{Objetivo}
\label{sec:org4afcd5b}
El alumno aplicará los conocimientos adquiridos en clase para resolver ejercicios de problemas aritméticos en el sistema binario.

\section*{Equipo}
\label{sec:org05d84d8}
Calculadora con capacidad de realizar operaciones aritméticas binarias.
\pagebreak

\section*{Desarrollo}
\label{sec:org908f269}
\subsection*{1. Realice las siguientes sumas binarias (8 bits). Muestre el procedimiento utilizado:}
\label{sec:orge383424}
\subsubsection*{a. 35 + 10}
\label{sec:org5637fd1}
\begin{verbatim}
              1
   35:  0010'0011
+  10:  0000'1010
  ----------------
   45:  0010'1101
\end{verbatim}

\subsubsection*{b. 23 + 17}
\label{sec:org9e3218b}
\begin{verbatim}
          11 111
   23:  0001'0111 
+  27:  0001'1011 
  ------------------
   50:  0011'0010
\end{verbatim}

\subsubsection*{c. -54 + 36}
\label{sec:org843ac33}
\begin{verbatim}
|---------+-----------+---------------+---------------|
| Decimal | Binario   | Complemento 1 | Complemento 2 |
|---------+-----------+---------------+---------------|
|      54 | 0011'0110 | 1100'1001     | 1100'1010     |
|      18 | 0001'0010 | 1110'1101     | 1110'1110     |
|---------+-----------+---------------+---------------| 

  -54:  1100'1010
+  36:  0010'0100
  ------------------
  -18:  1110 1110
\end{verbatim}

\subsubsection*{d. 78 + 50}
\label{sec:org826539b}
\begin{verbatim}
        1111 11
   78:  0100'1110 
+  50:  0011'0010 
  ------------------
  128:  1000'0000
\end{verbatim}

\subsection*{2. Realice las siguientes restas binarias con cifras con signo representadas con la representación complemento a 2 (8 bits). Muestre el procedimiento utilizado:}
\label{sec:orgc1b3ebc}
\subsubsection*{a. -75 - 25}
\label{sec:orgdd6d5fb}
\begin{verbatim}
|---------+-----------+----------------+----------------|
| Decimal | Binario   | Complemento 1  | Complemento 2  |
|---------+-----------+----------------+----------------|
|      75 | 0100'1011 | 1011'0100      | 1011'0101      |
|      25 | 0001'1001 | 1110'0110      | 1110'0111      |
|---------+-----------+----------------+----------------|

|---------------+---------------+-------------+---------|
| Complemento 2 | Complemento 1 | Binario     | Decimal |
|---------------+---------------+-------------+---------|
| 1'1001'1100   | 1'1001'1011   | 0 0110 0100 |     100 |
|               |               |             |         |
|---------------+---------------+-------------+---------|

           Overflow                  No Overflow

              11   111                     02 2    
       -75:   1011'0101           -75:   1011'0101             
    +  -25:   1110'0111        -   25:   0001'1001 
      ------------------         ------------------    
      -100: 1'1001'1100          -100:   1001'1100
\end{verbatim}

\subsubsection*{b. 78 - 50}
\label{sec:org29eb33f}
\begin{verbatim}
|---------+-----------+---------------+---------------|
| Decimal | Binario   | Complemento 1 | Complemento 2 |
|---------+-----------+---------------+---------------|
|      78 | 0011'0010 |     1100'1101 |      11001110 |
|---------+-----------+---------------+---------------|

              1  1 11
        78:   0100'1110    
    +  -50:   1100'1110
      ------------------      
        28: 1'0001'1100
\end{verbatim}

\subsubsection*{c. -92 – 40}
\label{sec:org77f5fca}
\begin{verbatim}
|---------+-----------+---------------+---------------|
| Decimal | Binario   | Complemento 1 | Complemento 2 |
|---------+-----------+---------------+---------------|
|      92 | 0101'1100 | 1010'0011     | 1010'0100     |
|      40 | 0010'1000 | 1101'0111     | 1101'1000     |
|---------+-----------+---------------+---------------| 

|---------------+---------------+-----------+---------|
| Complemento 2 | Complemento 1 | Binario   | Decimal |
|---------------+---------------+-----------+---------|
| 1'0111'1100   | 1'0111'1011   | 1000'0100 |     132 |
|---------------+---------------+-----------+---------|

              1       
       -92:   1010'0100
    +  -40:   1101'1000 
      ------------------      
      -132: 1'0111'1100
\end{verbatim}

\subsubsection*{d. 62 – 36}
\label{sec:orgad6cd06}
\begin{verbatim}
|---------+-----------+---------------+---------------|
| Decimal | Binario   | Complemento 1 | Complemento 2 |
|---------+-----------+---------------+---------------|
|      36 | 0010'0100 |     1101'1011 |     1101'1100 |
|---------+-----------+---------------+---------------|

              1111 1
        62:   0011'1110 
    +  -36:   1101'1100
      ------------------      
        26: 1'0001'1010
\end{verbatim}

\subsection*{3. Realice las siguientes operaciones aritméticas sobre números hexadecimales., Muestre el procedimiento utilizado:}
\label{sec:orgd5b388d}
\subsubsection*{a. 12h + 78h}
\label{sec:orgbeb306f}
\begin{verbatim}
   18:  12
+ 120:  78
 ----------
  138:  8A
\end{verbatim}

\subsubsection*{b. F5h – D8h}
\label{sec:org4eb277a}
\begin{verbatim}
  245:  F5
+ 216:  D8
 ----------
  461: 1CD
\end{verbatim}

\subsection*{4. Realice las siguientes multiplicaciones y divisiones con números binarios.}
\label{sec:org60234b7}
\subsubsection*{a. 25*3}
\label{sec:orga1df99a}
\begin{verbatim}
25:  0001 1001
 3:  0000 0011
--------------
     0001 1001
...0 0011 001.
---------------
75:  0100 1011  
\end{verbatim}

\subsubsection*{b. 75/5}
\label{sec:orgc86a666}
\begin{verbatim}
         1111                 02           0222
    +------------             1001         1000
101 | 1001011               -  101       -  101  
       101                   ------       ------    
    -------------              100         0011
       1000
        101
    -------------
        0111                  1
         101                  2
    -------------             4
         0101              +  8
          101               ----
    -------------            15 
          000
\end{verbatim}

\subsection*{5. Realice las siguientes conversiones de decimal a representación de punto flotante en precisión simple.}
\label{sec:org996e8c7}
\subsubsection*{a. 200.09375}
\label{sec:orgced60d5}
\begin{itemize}
\item Decimal a binario
\label{sec:org94d7b76}
\begin{verbatim}
200 = 11001000

0.09375 * 2 = 0.1875 | 0
0.1875  * 2 = 0.375  | 0
0.75    * 2 = 1.5    | 1
0.5     * 2 = 1      | 1

0.09375 = .0011

-------------------------

200.09375 = 11001000.0011
\end{verbatim}

\item Desplazamiento a la izquierda
\label{sec:org3203a6b}
\begin{verbatim}
1 < 1100100.00011
2 < 110010.000011
3 < 11001.0000011
4 < 1100.10000011
5 < 110.010000011
6 < 11.0010000011
7 < 1.10010000011
\end{verbatim}

\item bias:
\label{sec:orgbd4d9a5}
7 + 127 = 134

\item exponente:
\label{sec:org2787ed3}
bin(134) = 10000110 

\item matiza:
\label{sec:orgadf205b}
1.10010000011 = [ 10010000011 ]

\item matiza decimal
\label{sec:org2429610}
\begin{center}
\begin{tabular}{rrrrrrrrrrrr}
\hline
2\(^{\text{0}}\) & 2\(^{\text{-1}}\) & 2\(^{\text{-2}}\) & 2\(^{\text{-3}}\) & 2\(^{\text{-4}}\) & 2\(^{\text{-5}}\) & 2\(^{\text{-6}}\) & 2\(^{\text{-7}}\) & 2\(^{\text{-8}}\) & 2\(^{\text{-9}}\) & 2\(^{\text{-10}}\) & 2\(^{\text{-11}}\)\\
\hline
1 & 1 & 0 & 0 & 1 & 0 & 0 & 0 & 0 & 0 & 1 & 1\\
\hline
\end{tabular}
\end{center}

\begin{center}
\begin{tabular}{l}
1 + 2\(^{\text{-1}}\) + 2\(^{\text{-4}}\) + 2\(^{\text{-10}}\) + 2\(^{\text{-11}}\) = 1.56396484375\\
\end{tabular}
\end{center}

\item binario
\label{sec:orgf81b226}
\begin{center}
\begin{tabular}{rll}
\hline
signo & exponente & matiza\\
\hline
0 & 1000 0110 & 1001 0000 0011 0000 0000 000\\
\hline
\end{tabular}
\end{center}
\end{itemize}

\subsection*{6. Realice las siguientes conversiones de representación de punto flotante en precisión simple a decimal.}
\label{sec:org79f6262}
\subsubsection*{a. 10111100010001110001110000000000}
\label{sec:orgd6e3535}
\begin{center}
\begin{tabular}{rrr}
\hline
signo & Exponente & Matiza\\
\hline
1 & 01111000 & 10001110001110000000000\\
\hline
\end{tabular}
\end{center}

\begin{itemize}
\item Exponente
\label{sec:org7628afd}
\begin{verbatim}
01111000 = 120

 x + 127 = 120 
       x = 120 - 127
       x = -7
\end{verbatim}

\item Matiza
\label{sec:org384fca2}
\begin{verbatim}
matiza = "10001110001110000000000"
a = [ 2**(-(i+1)) for i,v in enumerate(matiza) if v=='1' ]
return sum(a)
\end{verbatim}

\begin{verbatim}
0.5555419921875
\end{verbatim}

\item Conversión
\label{sec:orgdd219f5}
\[
-1 \times 2^{-7} \times (1+0.5555419921875) = -0.012152671813964844
\]
\end{itemize}

\subsection*{7. Realice las siguientes conversiones de BCD a Decimal y de Decimal a BCD., Muestre el procedimiento utilizado:}
\label{sec:org8ec5b44}
\subsubsection*{a. 1001-0111-1000-0001}
\label{sec:orgf352e43}
\begin{verbatim}
1001-0111-1000-0001
   9    7    8    1
\end{verbatim}

\subsubsection*{b. 0010-1011-1000-0011}
\label{sec:org677868b}
\begin{verbatim}
0010-1011-1000-0011
   2   11    8    3
\end{verbatim}

\begin{description}
\item[{El BCD es incorrecto}] para que un valor sea valido tiene que estar en el rango [0-9], '11' esta fuera del rango, por lo que no puede representar el numero en decimal correspondiente.
\end{description}

\subsubsection*{c. 9578}
\label{sec:orgca8f7b1}
\begin{verbatim}
   9    5    7    8
1001-0101-0111-1000 
\end{verbatim}

\subsubsection*{d. 2136}
\label{sec:org0e9719d}
\begin{verbatim}
   2    1    3    6
0010-0001-0011-0110
\end{verbatim}

\section*{Conclusiones y comentarios}
\label{sec:orgb281da2}
Las diferentes representaciones de los números son muy útiles para representar entidades mas complejas o cosas donde otra propiedad del objeto es mas importante que su valor, por ejemplo un bitfield es muy complicado de escribir de manera manual pero como se puede representar el binario ya que lo único que nos interesa es su posición:

\subsection*{Ejmplo}
\label{sec:orgbbfa9c8}
\begin{verbatim}
  16  8   4   2   1   DEC 
|---+---+---+---+---+-----|
| .   .   .   .   . |   0 |
|---+---+---+---+---+-----|
| .   1   .   1   . |  10 |
|---+---+---+---+---+-----|
| .   .   .   .   . |   0 |
|---+---+---+---+---+-----|
| 1   .   .   .   1 |  17 |
|---+---+---+---+---+-----|
| .   1   1   1   . |  14 |
|---+---+---+---+---+-----|
| .   .   .   .   . |   0 |
|---+---+---+---+---+-----|

Toda esta matriz binaria se puede representar como: [0,A,0,11,E,0]
\end{verbatim}
\end{document}
