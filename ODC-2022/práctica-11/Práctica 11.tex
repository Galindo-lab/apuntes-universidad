% Created 2022-11-18 vie 09:40
% Intended LaTeX compiler: pdflatex
\documentclass[12pt]{article}
\usepackage[utf8]{inputenc}
\usepackage[T1]{fontenc}
\usepackage{graphicx}
\usepackage{grffile}
\usepackage{longtable}
\usepackage{wrapfig}
\usepackage{rotating}
\usepackage[normalem]{ulem}
\usepackage{amsmath}
\usepackage{textcomp}
\usepackage{amssymb}
\usepackage{capt-of}
\usepackage{hyperref}
\usepackage[spanish]{babel}
\usepackage{graphicx,geometry}
\geometry{ a4paper, left=1in, right=1in, top=1in, bottom=1in }
\renewcommand\familydefault{\sfdefault}
\usepackage{sectsty}
\sectionfont{\normalfont\Large }
\subsectionfont{\normalfont}
\usepackage{tabularx}
\usepackage{listings}
\lstdefinestyle{mystyle}{
numbers=left,
showspaces=false,
frame=single,
showspaces=false,
showstringspaces=false,
showtabs=false,
numberstyle=\tiny,
aboveskip=\parskip
}
\lstset{
style=mystyle,
literate={á}{{\'a}}1
{é}{{\'e}}1
{í}{{\'{\i}}}1
{ó}{{\'o}}1
{ú}{{\'u}}1
{Á}{{\'A}}1
{É}{{\'E}}1
{Í}{{\'I}}1
{Ó}{{\'O}}1
{Ú}{{\'U}}1
{ü}{{\"u}}1
{Ü}{{\"U}}1
{ñ}{{\~n}}1
{Ñ}{{\~N}}1
{¿}{{?``}}1
{¡}{{!``}}1
}
\makeatletter
\usepackage{fancyhdr}
\pagestyle{fancy}
\usepackage{mdframed}
\BeforeBeginEnvironment{minted}{\begin{mdframed}}
\AfterEndEnvironment{minted}{\end{mdframed}}
\author{Luis Eduardo Galindo Amaya (1274895)}
\date{18-11-2022}
\title{Macros y Lenguaje Ensamblador de Alto Nivel}
\hypersetup{
 pdfauthor={Luis Eduardo Galindo Amaya (1274895)},
 pdftitle={Macros y Lenguaje Ensamblador de Alto Nivel},
 pdfkeywords={},
 pdfsubject={},
 pdfcreator={Emacs 26.3 (Org mode 9.1.9)}, 
 pdflang={Spanish}}
\begin{document}



\newcommand{\docente}{Arturo Arreola Alvarez}
\newcommand{\asignatura}{Organización de Computadoras (331)}
\newcommand{\semestre}{2022-2}

\newcommand{\miportada}[1]{
	\begin{titlepage}
		\vspace*{0.75in}
		\begin{flushleft}
			\sffamily
			\large #1       \\
			\Huge 
            \@title         \\
			\hrulefill
			\vspace{0.25in} \\
			\Large \@author \\
			\vspace*{\fill}
            \includegraphics[width=\textwidth]{../includes/filler.png} \\
			\vspace*{\fill}
			\large
			\begin{tabular}{|l|l|}
              \hline
			  Asignatura & \asignatura \\
			  Docente    & \docente    \\
			  Fecha      & \@date      \\
              \hline
			\end{tabular}
		\end{flushleft}
	\end{titlepage}
}

\miportada{ Práctica 11 }

\fancyhf{}
\lhead{ \asignatura }
\rhead{ \semestre }
\rfoot{Página \thepage}

\setlength\parindent{0pt}   % eliminar el intentado
\setlength{\parskip}{1.2em}
\maketitle

\section*{Objetivo}
\label{sec:orgaedb11d}
Identificar el funcionamiento y beneficio de utilizar macros en un lenguaje de bajo nivel. Identificar la estructura del  lenguaje ensamblador de alto nivel para el desarrollo de sistemas basados en microprocesador.

\section*{Desarollo}
\label{sec:org90fe00b}
Cree un archivo llamado P11\_Menu.asm en donde deberá escribir un programa en lenguaje ensamblador que haga lo siguiente:

\begin{enumerate}
\item El programa deberá mostrar un menú principal con al menos 2 opciones y una opción para terminar el programa.

\item Cree la macro print\_menu que le ayudara a la impresión de los menús.

\item Cuando el usuario seleccione una de las opciones se debe desplegar un submenú con al menos 2 opciones y una opción para regresar al menú principal.
\end{enumerate}

\begin{verbatim}
Ejemplo:
Menú
|Operaciones Aritméticas
 |Suma
 |Resta
 |Regresar
|Operaciones Binarias
  |AND
  |OR
  |Regresar
|Salir
\end{verbatim}

\section*{Conclusiones y comentarios}
\label{sec:org14773e2}
Cuando trabajamos con macros debemos ser cuidadosos, pienso que no debemos abusar de ellos, en mi opinion un macro adecuado no debe tener mas de 10 lineas de codigo y no ser abusado exesivamente.


\section*{Código}
\label{sec:orgdf46bad}
\\ \lstinputlisting{./src/P11.asm}
\end{document}
