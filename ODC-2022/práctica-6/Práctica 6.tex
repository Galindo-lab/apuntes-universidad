% Created 2022-09-27 mar 11:52
% Intended LaTeX compiler: pdflatex
\documentclass[12pt]{article}
\usepackage[utf8]{inputenc}
\usepackage[T1]{fontenc}
\usepackage{graphicx}
\usepackage{grffile}
\usepackage{longtable}
\usepackage{wrapfig}
\usepackage{rotating}
\usepackage[normalem]{ulem}
\usepackage{amsmath}
\usepackage{textcomp}
\usepackage{amssymb}
\usepackage{capt-of}
\usepackage{hyperref}
\usepackage[spanish]{babel}
\usepackage{graphicx,geometry}
\geometry{ a4paper, left=1in, right=1in, top=1in, bottom=1in }
\renewcommand\familydefault{\sfdefault}
\usepackage{sectsty}
\sectionfont{\normalfont\Large }
\subsectionfont{\normalfont}
\usepackage{tabularx}
\usepackage{listings}
\lstdefinestyle{mystyle}{
numbers=left,
showspaces=false,
frame=leftline,
showspaces=false,
showstringspaces=false,
showtabs=false,
numberstyle=\tiny,
}
\lstset{
style=mystyle,
literate={á}{{\'a}}1
{é}{{\'e}}1
{í}{{\'{\i}}}1
{ó}{{\'o}}1
{ú}{{\'u}}1
{Á}{{\'A}}1
{É}{{\'E}}1
{Í}{{\'I}}1
{Ó}{{\'O}}1
{Ú}{{\'U}}1
{ü}{{\"u}}1
{Ü}{{\"U}}1
{ñ}{{\~n}}1
{Ñ}{{\~N}}1
{¿}{{?``}}1
{¡}{{!``}}1
}
\makeatletter
\usepackage{fancyhdr}
\pagestyle{fancy}
\usepackage{mdframed}
\BeforeBeginEnvironment{minted}{\begin{mdframed}}
\AfterEndEnvironment{minted}{\end{mdframed}}
\author{Luis Eduardo Galindo Amaya (1274895)}
\date{23-09-2022}
\title{Modos de Direccionamiento}
\hypersetup{
 pdfauthor={Luis Eduardo Galindo Amaya (1274895)},
 pdftitle={Modos de Direccionamiento},
 pdfkeywords={},
 pdfsubject={},
 pdfcreator={Emacs 26.3 (Org mode 9.1.9)}, 
 pdflang={Spanish}}
\begin{document}



\newcommand{\docente}{Arturo Arreola Alvarez}
\newcommand{\asignatura}{Organización de Computadoras (331)}
\newcommand{\semestre}{2022-2}

\newcommand{\miportada}[1]{
	\begin{titlepage}
		\vspace*{0.75in}
		\begin{flushleft}
			\sffamily
			\large #1       \\
			\Huge 
            \@title         \\
			\hrulefill
			\vspace{0.25in} \\
			\Large \@author \\
			\vspace*{\fill}
            \includegraphics[width=\textwidth]{../includes/filler.png} \\
			\vspace*{\fill}
			\large
			\begin{tabular}{|l|l|}
              \hline
			  Asignatura & \asignatura \\
			  Docente    & \docente    \\
			  Fecha      & \@date      \\
              \hline
			\end{tabular}
		\end{flushleft}
	\end{titlepage}
}

\miportada{ Práctica 6 }

\fancyhf{}
\lhead{ \asignatura }
\rhead{ \semestre }
\rfoot{Página \thepage}

\setlength\parindent{0pt}   % eliminar el intentado
\setlength{\parskip}{1.2em}
\maketitle

\section*{Objetivo}
\label{sec:org69464ad}
Identificar los modos de direccionamiento adecuados para manejo de memoria en aplicaciones de sistemas basados en microprocesador mediante la distinción de su funcionamiento, de forma lógica y responsable.

\section*{Desarollo}
\label{sec:org3b0ad69}
\subsection*{Actividad 1}
\label{sec:orgc795b95}
El programa solicita al usuario el ingreso de su nombre y despliega un mensaje de saludo.

\subsubsection*{Código}
\label{sec:orgf6ffd43}
\\ \lstinputlisting{./src/captura.asm}


\subsection*{Actividad 2}
\label{sec:org9bd64ca}
Cree un programa llamado Apellido\(_{\text{Nombre}}\)\(_{\text{P6.asm}}\) que contenga las instrucciones necesarias para hacer lo que se indica a continuación:

\begin{description}
\item[{a)}] Reservar dos espacios en memoria no inicializados, uno de 32 bytes etiquetado como A y el otro de 1 byte etiquetado como N.

\item[{b)}] Solicitar una cadena que se almacene en A.

\item[{c)}] Copiar el caracter en la posición 0 de A en la variable N. Use un modo de direccionamiento base.

\item[{d)}] Reemplazar el caracter en la posición 3 de A por un asterisco ‘*’, usando un modo de direccionamiento base con desplazamiento.

\item[{e)}] Reemplazar el caracter en la posición 6 de A por un arroba ‘@’ usando un direccionamiento base con índice escalado.

\item[{f)}] Copiar el caracter en la posición 1 de A y almacenarlo en los bits 15-8 del acumulador.

\item[{g)}] Reemplazar el caracter en la posición 9 de A por el caracter en los bits 15-8 del acumulador, usando un direccionamiento base con índice escalado y desplazamiento.

\item[{h)}] Solicite un caracter al usuario y guárdelo en la posición 5 de A.
\end{description}


\subsubsection*{Código}
\label{sec:org8171aa6}
\\ \lstinputlisting{./src/t1.asm}

\section*{Conclusiones y comentarios}
\label{sec:org26458d7}
\begin{mdframed}
Es interesante ver que hay mas maneras de redireccionar memoria, a pesar de que se ve un poco extraño redireccionar con índice escalado esto nos permite acceder mas rápidamente a las posiciones de memoria y es importante aprender a usarlo.
\end{mdframed}

\section*{Dificultades en el desarrollo}
\label{sec:org4f36b2f}
\begin{mdframed}
En el inciso 'h' del ejercicio 2 al insertar el valor no mostraba el resultado final insertaba un salto de linea, tuve que pasar el valor capturado al valor 'N' para eliminar el salto de linea.
\end{mdframed}
\end{document}
