% Created 2021-12-07 mar 19:42
% Intended LaTeX compiler: pdflatex
\documentclass{article}
\usepackage[utf8]{inputenc}
\usepackage[T1]{fontenc}
\usepackage{graphicx}
\usepackage{grffile}
\usepackage{longtable}
\usepackage{wrapfig}
\usepackage{rotating}
\usepackage[normalem]{ulem}
\usepackage{amsmath}
\usepackage{textcomp}
\usepackage{amssymb}
\usepackage{capt-of}
\usepackage{hyperref}
\usepackage[spanish]{babel}
\usepackage{pifont}
\usepackage{pagecolor,lipsum}
\pagecolor{pink!90!pink}
\author{Luis Eduardo Galindo Amaya}
\date{\today}
\title{Programas Cálculo}
\hypersetup{
 pdfauthor={Luis Eduardo Galindo Amaya},
 pdftitle={Programas Cálculo},
 pdfkeywords={},
 pdfsubject={},
 pdfcreator={Emacs 26.3 (Org mode 9.1.9)}, 
 pdflang={Spanish}}
\begin{document}

\maketitle
\tableofcontents

\newpage 

\section{Conversión Entre Sistemas De Coordenadas}
\label{sec:org919699c}
\subsection{Rectangulares a Cilíndricas (o Polares)}
\label{sec:org551f49d}
\begin{verbatim}
# Sustituye el valor de 'x', 'y' y 'z'.

x = 4
y = -5
z = 2


r = sqrt(x^2+y^2)
theta = arctan(y/x)

# determinar la cantidad de ángulo faltante
# 'and' regresa 1 = True y 0 = False
ajuste(x,y) = ( 
   and(x>=0,y>=0) * 0     + # I
   and(x<0 ,y>=0) * pi    + # II
   and(x<0 ,y<0)  * pi    + # III
   and(x>=0,y<0)  * 2*pi    # IV
)

# sumamos los grados faltantes
theta = theta+ajuste(x,y)

"Rectangular (x,y,z):"
float((x,y,z))

"Cilíndrica (r,theta,z):"
float((r,theta,z))
\end{verbatim}

\noindent\rule{\textwidth}{0.5pt}

\newpage 

\subsection{Rectangulares a Esféricas}
\label{sec:org7e8517c}
\begin{verbatim}
# Sustituye el valor de 'x', 'y' y 'z'.

x = 4
y = -5
z = 2


rho = sqrt(x^2+y^2+z^2)
theta = arctan(y/x)
phi = arccos(z/rho)

# determinar la cantidad de ángulo faltante
# 'and' regresa 1 = True y 0 = False
ajuste(x,y) = ( 
   and(x>=0,y>=0) * 0     + # I
   and(x<0 ,y>=0) * pi    + # II
   and(x<0 ,y<0)  * pi    + # III
   and(x>=0,y<0)  * 2*pi    # IV
)

# sumamos los grados faltantes
theta = theta+ajuste(x,y)

"Rectangular (x,y,z):"
float((x,y,z))

"Esféricas (rho,theta,phi):"
float((rho,theta,phi))
\end{verbatim}

\noindent\rule{\textwidth}{0.5pt}

\newpage 

\subsection{Cilíndricas a Rectangulares}
\label{sec:org3a59829}
\begin{verbatim}
# Sustituye el valor de 'r', 'theta' y 'z'.

r = 4
theta = 2
z = 4


x = r * cos(theta)
y = r * sin(theta)
z = z

"Cilíndrica (r,theta,z):"
float((r,theta,z))

"Rectangular (x,y,z):"
float((x,y,z))
\end{verbatim}

\noindent\rule{\textwidth}{0.5pt}

\newpage 

\subsection{Cilíndricas a Esféricas}
\label{sec:org87e1f40}
\begin{verbatim}
# Sustituye el valor de 'r', 'theta' y 'z'
# theta es el angulo de los ejes 'x' y 'y'

r = 1
theta = 1
z = 1


rho = sqrt(r^2+z^2) 
theta = theta
phi = arccos(z/rho)

"Cilindrica (r,theta,z):"
float((r,theta,z))

"Esferica (rho,theta,phi):"
float((rho,theta,phi))
\end{verbatim}

\noindent\rule{\textwidth}{0.5pt}

\newpage

\subsection{Esfericas a Rectangulares}
\label{sec:org4523282}
\begin{verbatim}
# Sustituye el valor de 'rho', 'theta' y 'phi'
# theta es el angulo de los ejes 'x' y 'y'
# phi es el angulo del eje 'z'

rho = 1
theta = 1
phi = 1


x = rho * sin(phi) * cos(theta)
y = rho * sin(phi) * sin(theta)
z = rho * cos(phi)

"Esferica (rho,theta,phi):"
float((rho,theta,phi))

"Rectangular (x,y,z):"
float((x,y,z))
\end{verbatim}

\noindent\rule{\textwidth}{0.5pt}

\newpage 

\subsection{Esfericas a Cilidnricas}
\label{sec:orgb024d8d}
\begin{verbatim}
# Sustituye el valor de 'rho', 'theta' y 'phi'
# theta es el angulo de los ejes 'x' y 'y'
# phi es el angulo del eje 'z'

rho = 1
theta = 1
phi = 1


r = rho * sin(phi)
theta = theta
z = rho * cos(phi)

"Esferica (rho,theta,phi):"
float((rho,theta,phi))

"Cilindrica (r,theta,z):"
float((r,theta,z))
\end{verbatim}

\noindent\rule{\textwidth}{0.5pt}

\newpage

\section{Modulo del Vector}
\label{sec:org4c3fcdb}
\subsection{Modulo - Version 1}
\label{sec:orgbce2768}
\begin{verbatim}
# Sustituye el valor de 'x', 'y' y 'z'.

x = 1
y = 1
z = 1


sqrt(x^2 + y^2 + z^2)
\end{verbatim}

\noindent\rule{\textwidth}{0.5pt}

\subsection{Modulo - Version 2}
\label{sec:orgd8f5923}
\begin{verbatim}
# Sustituye los valores por los de tu vector (x,y,z).

v = (1,3,5)


abs(v)
\end{verbatim}

\noindent\rule{\textwidth}{0.5pt}

\subsection{Modulo del Vector Fuera Del Origen}
\label{sec:org9a3e4ea}
\begin{verbatim}
# Sustituye 'v' por los valores por los de tu vector.
# Sustituye 'g' los valores por los de el origen.

v = (1,3,5) # Vector
g = (0,0,0) # Origen


abs(v-g)
\end{verbatim}

\noindent\rule{\textwidth}{0.5pt}
\end{document}
