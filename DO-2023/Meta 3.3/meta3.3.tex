% Created 2023-10-27 vie 14:24
% Intended LaTeX compiler: pdflatex
\documentclass[11pt]{article}
\usepackage[utf8]{inputenc}
\usepackage[T1]{fontenc}
\usepackage{graphicx}
\usepackage{grffile}
\usepackage{longtable}
\usepackage{wrapfig}
\usepackage{rotating}
\usepackage[normalem]{ulem}
\usepackage{amsmath}
\usepackage{textcomp}
\usepackage{amssymb}
\usepackage{capt-of}
\usepackage{hyperref}
\author{Luis Eduardo Galindo Amaya (1274895)}
\date{jueves, 26 octubre 2023}
\title{Matriz de capacidades tecnológicas y multiculturalidad}
\hypersetup{
 pdfauthor={Luis Eduardo Galindo Amaya (1274895)},
 pdftitle={Matriz de capacidades tecnológicas y multiculturalidad},
 pdfkeywords={},
 pdfsubject={},
 pdfcreator={Emacs 27.1 (Org mode 9.3)}, 
 pdflang={Spanish}}
\begin{document}



\section*{Ensayo}
\label{sec:org43683c1}
La multiculturalidad en las organizaciones se ha convertido en una fuente de 
ventaja competitiva. Este ensayo explora cómo la diversidad cultural influye en 
la innovación y el acceso a mercados globales.


La multiculturalidad promueve la innovación al fomentar la diversidad de
pensamiento y la generación de ideas creativas. Equipos multiculturales son más
propensos a encontrar soluciones originales a los desafíos empresariales.


La diversidad cultural facilita la entrada a mercados internacionales. Empleados
que comprenden la cultura del mercado objetivo pueden ayudar a la organización
a establecerse con éxito. Además, la adaptación a las diferencias culturales 
se vuelve más efectiva con una fuerza laboral diversa.
\end{document}
