% Created 2022-09-06 mar 11:07
% Intended LaTeX compiler: pdflatex
\documentclass[12pt]{article}
\usepackage[utf8]{inputenc}
\usepackage[T1]{fontenc}
\usepackage{graphicx}
\usepackage{grffile}
\usepackage{longtable}
\usepackage{wrapfig}
\usepackage{rotating}
\usepackage[normalem]{ulem}
\usepackage{amsmath}
\usepackage{textcomp}
\usepackage{amssymb}
\usepackage{capt-of}
\usepackage{hyperref}
\usepackage[spanish]{babel}
\usepackage{graphicx,geometry}
\geometry{ a4paper, left=1in, right=1in, top=1in, bottom=1in }
\renewcommand\familydefault{\sfdefault}
\usepackage{sectsty}
\sectionfont{\normalfont\Large }
\subsectionfont{\normalfont}
\usepackage{tabularx}
\usepackage{listings}
\lstdefinestyle{mystyle}{
numbers=left,
showspaces=false,
frame=leftline,
showspaces=false,
showstringspaces=false,
showtabs=false,
numberstyle=\tiny,
}
\lstset{
style=mystyle,
literate={á}{{\'a}}1
{é}{{\'e}}1
{í}{{\'{\i}}}1
{ó}{{\'o}}1
{ú}{{\'u}}1
{Á}{{\'A}}1
{É}{{\'E}}1
{Í}{{\'I}}1
{Ó}{{\'O}}1
{Ú}{{\'U}}1
{ü}{{\"u}}1
{Ü}{{\"U}}1
{ñ}{{\~n}}1
{Ñ}{{\~N}}1
{¿}{{?``}}1
{¡}{{!``}}1
}
\makeatletter
\usepackage{fancyhdr}
\pagestyle{fancy}
\usepackage{mdframed}
\BeforeBeginEnvironment{minted}{\begin{mdframed}}
\AfterEndEnvironment{minted}{\end{mdframed}}
\author{Luis Eduardo Galindo Amaya (1274895)}
\date{2022-09-05}
\title{Sistemas Numéricos}
\hypersetup{
 pdfauthor={Luis Eduardo Galindo Amaya (1274895)},
 pdftitle={Sistemas Numéricos},
 pdfkeywords={},
 pdfsubject={},
 pdfcreator={Emacs 26.3 (Org mode 9.1.9)}, 
 pdflang={Spanish}}
\begin{document}



\newcommand{\docente}{Arturo Arreola Alvarez}
\newcommand{\asignatura}{Organización de Computadoras (331)}
\newcommand{\semestre}{2022-2}

\newcommand{\miportada}[1]{
	\begin{titlepage}
		\vspace*{0.75in}
		\begin{flushleft}
			\sffamily
			\large #1       \\
			\Huge 
            \@title         \\
			\hrulefill
			\vspace{0.25in} \\
			\Large \@author \\
			\vspace*{\fill}
            \includegraphics[width=\textwidth]{../includes/filler.png} \\
			\vspace*{\fill}
			\large
			\begin{tabular}{|l|l|}
              \hline
			  Asignatura & \asignatura \\
			  Docente    & \docente    \\
			  Fecha      & \@date      \\
              \hline
			\end{tabular}
		\end{flushleft}
	\end{titlepage}
}

\miportada{ Práctica 3 }

\fancyhf{}
\lhead{ \asignatura }
\rhead{ \semestre }
\rfoot{Página \thepage}

\setlength\parindent{0pt}   % eliminar el intentado
\setlength{\parskip}{1.2em}
\maketitle

\section*{Ejercicio 1}
\label{sec:orgfec3dd7}
Realice las siguientes conversiones de sistema decimal a sistema binario. Muestre el procedimiento utilizado:

\subsection*{a) 200}
\label{sec:orgb3348e7}
\begin{mdframed}
\begin{center}
\begin{tabular}{lrrrrrrrr}
n & 200 & 100 & 50 & 25 & 12 & 6 & 3 & 1\\
\hline
n/2 & 100 & 50 & 25 & 12 & 6 & 3 & 1 & 0\\
n\%2 & 0 & 0 & 0 & 1 & 0 & 0 & 1 & 1\\
\end{tabular}
\end{center}

\begin{center}
\begin{tabular}{l}
Resultado: 11001000\\
\end{tabular}
\end{center}
\end{mdframed}

\subsection*{b) 680}
\label{sec:org70e3bd8}
\begin{mdframed}
\begin{center}
\begin{tabular}{lrrrrrrrrrr}
n & 680 & 340 & 170 & 85 & 42 & 21 & 10 & 5 & 2 & 1\\
\hline
n/2 & 340 & 170 & 85 & 42 & 21 & 10 & 5 & 2 & 1 & 0\\
n\%2 & 0 & 0 & 0 & 1 & 0 & 1 & 0 & 1 & 0 & 1\\
\end{tabular}
\end{center}

\begin{center}
\begin{tabular}{l}
Resultado: 1010101000\\
\end{tabular}
\end{center}
\end{mdframed}

\subsection*{c) 76.375}
\label{sec:orgbfe72b2}
\begin{mdframed}
\textbf{Parte entera}
\begin{center}
\begin{tabular}{lrrrrrrr}
n & 76 & 38 & 19 & 9 & 4 & 2 & 1\\
\hline
n/2 & 38 & 19 & 9 & 4 & 2 & 1 & 0\\
n\%2 & 0 & 0 & 1 & 1 & 0 & 0 & 1\\
\end{tabular}
\end{center}

\textbf{Parte decimal}
\begin{center}
\begin{tabular}{lrrr}
n & .375 & .75 & .50\\
\hline
n*2 & 0.75 & 1.5 & 1\\
int & 0 & 1 & 1\\
\end{tabular}
\end{center}

\begin{center}
\begin{tabular}{l}
Resultado: 1001100.011\\
\end{tabular}
\end{center}
\end{mdframed}

\subsection*{d) 0.34375}
\label{sec:org78b1ced}
\begin{mdframed}
\begin{center}
\begin{tabular}{lrrrrr}
n & 0.34375 & 0.6875 & 0.375 & 0.75 & 0.5\\
\hline
n*2 & 0.6875 & 1.375 & 0.75 & 1.5 & 1\\
int & 0 & 1 & 0 & 1 & 1\\
\end{tabular}
\end{center}

\begin{center}
\begin{tabular}{l}
Resultado: 0.01011\\
\end{tabular}
\end{center}
\end{mdframed}

\section*{Ejercicio 2}
\label{sec:org15af2fb}
Realice las siguientes conversiones de sistema decimal a sistema hexadecimal. Muestre el procedimiento utilizado:

\subsection*{a) 1536}
\label{sec:org1e24b24}
\begin{mdframed}
\begin{center}
\begin{tabular}{lrrr}
n & 1536 & 96 & 6\\
\hline
n/16 & 96 & 6 & 0\\
n\%16 & 0 & 0 & 6\\
hex(n\%16) & 0 & 0 & 6\\
\end{tabular}
\end{center}

\begin{center}
\begin{tabular}{l}
Resultado: 600\\
\end{tabular}
\end{center}
\end{mdframed}

\subsection*{b) 5000}
\label{sec:org9933f56}
\begin{mdframed}
\begin{center}
\begin{tabular}{lrrrr}
n & 5000 & 312 & 19 & 1\\
\hline
n/16 & 312 & 19 & 1 & 0\\
n\%16 & 8 & 8 & 3 & 1\\
hex(n\%16) & 8 & 8 & 3 & 1\\
\end{tabular}
\end{center}

\begin{center}
\begin{tabular}{l}
Resultado: 1388\\
\end{tabular}
\end{center}
\end{mdframed}

\subsection*{c) 856}
\label{sec:orgd8ad4dd}
\begin{mdframed}
\begin{center}
\begin{tabular}{lrrr}
n & 856 & 53 & 3\\
\hline
n/16 & 53 & 3 & 0\\
n\%16 & 8 & 5 & 3\\
hex(n\%16) & 8 & 5 & 3\\
\end{tabular}
\end{center}

\begin{center}
\begin{tabular}{l}
Resultado: 358\\
\end{tabular}
\end{center}
\end{mdframed}

\subsection*{d) 128}
\label{sec:org7c05df6}
\begin{mdframed}
\begin{center}
\begin{tabular}{lrr}
n & 128 & 8\\
\hline
n/16 & 8 & 0\\
n\%16 & 0 & 8\\
hex(n\%16) & 0 & 8\\
\end{tabular}
\end{center}

\begin{center}
\begin{tabular}{l}
Resultado: 80\\
\end{tabular}
\end{center}
\end{mdframed}

\section*{Ejercicio 3}
\label{sec:orga74320b}
Realice las siguientes conversiones de sistema hexadecimal a sistema decimal. Muestre el procedimiento utilizado:

\subsection*{a) 0xB4}
\label{sec:org1970404}
\[
11 * 16 + 4 = 180
\]

\subsection*{b) 0x123}
\label{sec:org4da309f}
\[
1*16^2 + 2*16 + 3 = 291
\]

\subsection*{c) 0xA5A5}
\label{sec:org3b0bca2}
\[
10*16^3 + 5*16^2 + 10*16 + 5 = 42405
\]

\subsection*{d) F001}
\label{sec:orgc4591ab}
\[
15*16^3+0+0+1 = 61441
\]

\section*{Ejercicio 4}
\label{sec:org2725beb}
Realice las siguientes conversiones de sistema binario a sistema decimal. Muestre el procedimiento utilizado:

\subsection*{a) 10011010}
\label{sec:org92b53c6}
\begin{center}
\begin{tabular}{rrrrrrrr}
128 & 64 & 32 & 16 & 8 & 4 & 2 & 1\\
\hline
1 & 0 & 0 & 1 & 1 & 0 & 1 & 0\\
\end{tabular}
\end{center}

\[
128 + 6 + 8 + 2 = 154
\]

\subsection*{b) 00110111}
\label{sec:org9853cff}
\begin{center}
\begin{tabular}{rrrrrrrr}
128 & 64 & 32 & 16 & 8 & 4 & 2 & 1\\
\hline
0 & 0 & 1 & 1 & 0 & 1 & 1 & 1\\
\end{tabular}
\end{center}

\[
32 + 16 + 4 + 2 + 1 = 55
\]

\subsection*{c) 11100001}
\label{sec:org0f6ed5c}
\begin{center}
\begin{tabular}{rrrrrrrr}
128 & 64 & 32 & 16 & 8 & 4 & 2 & 1\\
\hline
1 & 1 & 1 & 0 & 0 & 0 & 0 & 1\\
\end{tabular}
\end{center}

\[
128 + 64 + 32 + 1  = 225
\]

\subsection*{d) 10111100}
\label{sec:org6b48412}
\begin{center}
\begin{tabular}{rrrrrrrr}
128 & 64 & 32 & 16 & 8 & 4 & 2 & 1\\
\hline
1 & 0 & 1 & 1 & 1 & 1 & 0 & 0\\
\end{tabular}
\end{center}

\[
128 + 32 + 16 + 8 + 4 = 188
\]

\section*{Ejercicio 5}
\label{sec:orgb451fd9}
Realice las siguientes conversiones de sistema hexadecimal a sistema binario. Muestre el procedimiento utilizado:

\subsection*{a) BA}
\label{sec:org8902887}
\begin{mdframed}
\begin{center}
\begin{tabular}{rr}
B & A\\
\hline
1011 & 1010\\
\end{tabular}
\end{center}

\begin{center}
\begin{tabular}{l}
Resultado: 10111010\\
\end{tabular}
\end{center}
\end{mdframed}

\subsection*{b) D5}
\label{sec:orgdb43444}
\begin{mdframed}
\begin{center}
\begin{tabular}{rr}
D & 5\\
\hline
1101 & 0101\\
\end{tabular}
\end{center}

\begin{center}
\begin{tabular}{l}
Resultado: 11010101\\
\end{tabular}
\end{center}
\end{mdframed}

\subsection*{c) 9E}
\label{sec:org1c394ff}
\begin{mdframed}
\begin{center}
\begin{tabular}{rr}
9 & E\\
\hline
1001 & 1110\\
\end{tabular}
\end{center}

\begin{center}
\begin{tabular}{l}
Resultado: 10011110\\
\end{tabular}
\end{center}
\end{mdframed}

\subsection*{d) 32D8}
\label{sec:orge316702}
\begin{mdframed}
\begin{center}
\begin{tabular}{rrrr}
3 & 2 & D & 8\\
\hline
0011 & 0010 & 1101 & 1000\\
\end{tabular}
\end{center}

\begin{center}
\begin{tabular}{l}
Resultado: 11001011011000\\
\end{tabular}
\end{center}
\end{mdframed}

\section*{Ejercicio 6}
\label{sec:org18da1a9}
Realice las siguientes conversiones de sistema binario a sistema hexadecimal. Muestre el procedimiento utilizado:

\subsection*{a) 10011010}
\label{sec:org4e94c7e}
\begin{center}
\begin{tabular}{rr}
1001 & 1010\\
\hline
9 & A\\
\end{tabular}
\end{center}

\subsection*{b) 00110111}
\label{sec:org78aceeb}
\begin{center}
\begin{tabular}{rr}
0011 & 0111\\
\hline
3 & 7\\
\end{tabular}
\end{center}

\subsection*{c) 01011000}
\label{sec:orgbc206e3}
\begin{center}
\begin{tabular}{rr}
0101 & 1000\\
\hline
5 & 8\\
\end{tabular}
\end{center}

\subsection*{d) 11100001}
\label{sec:orgd68dbad}
\begin{center}
\begin{tabular}{rr}
1110 & 0001\\
\hline
E & 1\\
\end{tabular}
\end{center}

\section*{Ejercicio 7}
\label{sec:orge93cfde}
Represente los siguientes números en su formato signo-magnitud (8 bits):

\subsection*{a) +58}
\label{sec:org5c3fdf7}
\begin{mdframed}
\begin{center}
\begin{tabular}{lrrrrrr}
n & 58 & 29 & 14 & 7 & 3 & 1\\
\hline
n/2 & 29 & 14 & 7 & 3 & 1 & 0\\
n\%2 & 0 & 1 & 0 & 1 & 1 & 1\\
\end{tabular}
\end{center}

\begin{center}
\begin{tabular}{rr}
signo & magnitud\\
\hline
0 & 0111010\\
\end{tabular}
\end{center}

\begin{center}
\begin{tabular}{l}
Resultado: 0011 1010\\
\end{tabular}
\end{center}
\end{mdframed}

\subsection*{b) +37}
\label{sec:orgcb50bef}
\begin{mdframed}
\begin{center}
\begin{tabular}{lrrrrrr}
n & 37 & 18 & 9 & 4 & 2 & 1\\
\hline
n/2 & 18 & 9 & 4 & 2 & 1 & 0\\
n\%2 & 1 & 0 & 1 & 0 & 0 & 1\\
\end{tabular}
\end{center}

\begin{center}
\begin{tabular}{rr}
signo & magnitud\\
\hline
0 & 0100101\\
\end{tabular}
\end{center}

\begin{center}
\begin{tabular}{l}
Resultado: 00100101\\
\end{tabular}
\end{center}
\end{mdframed}

\subsection*{c) -101}
\label{sec:orgcd871ba}
\begin{mdframed}
\begin{center}
\begin{tabular}{lrrrrrrr}
n & 101 & 50 & 25 & 12 & 6 & 3 & 1\\
\hline
n/2 & 50 & 25 & 12 & 6 & 3 & 1 & 0\\
n\%2 & 1 & 0 & 1 & 0 & 0 & 1 & 1\\
\end{tabular}
\end{center}

\begin{center}
\begin{tabular}{rr}
signo & magnitud\\
\hline
1 & 1100101\\
\end{tabular}
\end{center}

\begin{center}
\begin{tabular}{l}
Resultado: 11100101\\
\end{tabular}
\end{center}
\end{mdframed}

\subsection*{d) -68}
\label{sec:org83afeba}
\begin{mdframed}
\begin{center}
\begin{tabular}{lrrrrrrr}
n & 68 & 34 & 17 & 8 & 4 & 2 & 1\\
\hline
n/2 & 34 & 17 & 8 & 4 & 2 & 1 & 0\\
n\%2 & 0 & 0 & 1 & 0 & 0 & 0 & 1\\
\end{tabular}
\end{center}

\begin{center}
\begin{tabular}{rr}
signo & magnitud\\
\hline
1 & 1000100\\
\end{tabular}
\end{center}

\begin{center}
\begin{tabular}{l}
Resultado: 11000100\\
\end{tabular}
\end{center}
\end{mdframed}

\section*{Ejercicio 8}
\label{sec:org1030369}
Represente los siguientes números en su formato signo-magnitud (8 bits):

\subsection*{a) +128}
\label{sec:orgdc51419}
\begin{mdframed}


\begin{center}
\begin{tabular}{lrrrrrrrr}
n & 128 & 64 & 32 & 16 & 8 & 4 & 2 & 1\\
\hline
n/2 & 64 & 32 & 16 & 8 & 4 & 2 & 1 & 0\\
n\%2 & 0 & 0 & 0 & 0 & 0 & 0 & 0 & 1\\
\end{tabular}
\end{center}

\begin{center}
\begin{tabular}{lrl}
 & MSB & Numero\\
\hline
bin & 1 & 000'0000\\
cmp\(_{\text{1}}\) & 0 & 111'1111\\
cmp\(_{\text{2}}\) & 1 & 000'0000\\
\end{tabular}
\end{center}

\begin{center}
\begin{tabular}{l}
Resultado: 1000'0000\\
\end{tabular}
\end{center}

Es posible aplicar el complemento dos al numero, sin embrago la representacion de 128 requiere del bit mas significativo por lo que no es posible registrar el valor junto con su signo. 
\end{mdframed}

\subsection*{b) +24}
\label{sec:orgdc5c218}
\begin{mdframed}
\begin{center}
\begin{tabular}{lrrrrr}
n & 24 & 12 & 6 & 3 & 1\\
\hline
n/2 & 12 & 6 & 3 & 1 & 0\\
n\%2 & 0 & 0 & 0 & 1 & 1\\
\end{tabular}
\end{center}

\begin{center}
\begin{tabular}{lrl}
stp & MSB & Numero\\
\hline
bin & 0 & 001'1000\\
cmp\(_{\text{1}}\) & 1 & 110'0111\\
cmp\(_{\text{2}}\) & 1 & 110'1000\\
\end{tabular}
\end{center}

\begin{center}
\begin{tabular}{l}
Resultado: 1110'1000\\
\end{tabular}
\end{center}
\end{mdframed}

\subsection*{c) -53}
\label{sec:org76f3d8a}
\begin{mdframed}
\begin{center}
\begin{tabular}{lrrrrrr}
n & 53 & 26 & 13 & 6 & 3 & 1\\
\hline
n/2 & 26 & 13 & 6 & 3 & 1 & 0\\
n\%2 & 1 & 0 & 1 & 0 & 1 & 1\\
\end{tabular}
\end{center}

\begin{center}
\begin{tabular}{lrl}
stp & MSB & Numero\\
\hline
bin & 0 & 011'0101\\
cmp\(_{\text{1}}\) & 1 & 100'1010\\
cmp\(_{\text{2}}\) & 1 & 100'1011\\
\end{tabular}
\end{center}

\begin{center}
\begin{tabular}{l}
Resultado: 1100'1011\\
\end{tabular}
\end{center}
\end{mdframed}

\subsection*{d) -24}
\label{sec:orgea62864}
\begin{mdframed}
\begin{center}
\begin{tabular}{lrrrrr}
n & 24 & 12 & 6 & 3 & 1\\
\hline
n/2 & 12 & 6 & 3 & 1 & 0\\
n\%2 & 0 & 0 & 0 & 1 & 1\\
\end{tabular}
\end{center}

\begin{center}
\begin{tabular}{lrl}
stp & MSB & Numero\\
\hline
bin & 0 & 001'1000\\
cmp\(_{\text{1}}\) & 1 & 110'0111\\
cmp\(_{\text{2}}\) & 1 & 110'1000\\
\end{tabular}
\end{center}

\begin{center}
\begin{tabular}{l}
Resultado: 1110'1000\\
\end{tabular}
\end{center}
\end{mdframed}
\section*{Conclusiones y comentarios}
\label{sec:org6698b21}
Cada sistema numérico es util en diferentes contextos, por ejemplo la conversion de deciminal a hexadecimal o binario es muy tardada de hacer y requiere de muchas diviciones, por otro lado hexadecimal y binario con muy faciles de convetir, si conoce como representar el numero en binario solo hace esa conversion y las añade una tras de otra. 
\end{document}
