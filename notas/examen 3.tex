% Created 2022-12-01 jue 09:09
% Intended LaTeX compiler: pdflatex
\documentclass[10pt]{article}
\usepackage[utf8]{inputenc}
\usepackage[T1]{fontenc}
\usepackage{graphicx}
\usepackage{grffile}
\usepackage{longtable}
\usepackage{wrapfig}
\usepackage{rotating}
\usepackage[normalem]{ulem}
\usepackage{amsmath}
\usepackage{textcomp}
\usepackage{amssymb}
\usepackage{capt-of}
\usepackage{hyperref}
\usepackage[spanish]{babel}
\usepackage{graphicx,geometry}
\geometry{ a4paper, left=1in, right=1in, top=1in, bottom=1in }
\renewcommand\familydefault{\sfdefault}
\usepackage{sectsty}
\sectionfont{\normalfont\Large }
\subsectionfont{\normalfont}
\usepackage{tabularx}
\usepackage{listings}
\lstdefinestyle{mystyle}{
basicstyle=\ttfamily\footnotesize,
numbers=left,
showspaces=false,
frame=single,
showspaces=false,
showstringspaces=false,
showtabs=false,
numberstyle=\tiny,
aboveskip=\parskip
}
\lstset{
style=mystyle,
literate={á}{{\'a}}1
{é}{{\'e}}1
{í}{{\'{\i}}}1
{ó}{{\'o}}1
{ú}{{\'u}}1
{Á}{{\'A}}1
{É}{{\'E}}1
{Í}{{\'I}}1
{Ó}{{\'O}}1
{Ú}{{\'U}}1
{ü}{{\"u}}1
{Ü}{{\"U}}1
{ñ}{{\~n}}1
{Ñ}{{\~N}}1
{¿}{{?``}}1
{¡}{{!``}}1
}
\makeatletter
\usepackage{fancyhdr}
\pagestyle{fancy}
\usepackage{mdframed}
\BeforeBeginEnvironment{minted}{\begin{mdframed}}
\AfterEndEnvironment{minted}{\end{mdframed}}
\author{Luis Eduardo Galindo Amaya (1274895)}
\date{01-12-2022}
\title{Notas para el examen 3}
\hypersetup{
 pdfauthor={Luis Eduardo Galindo Amaya (1274895)},
 pdftitle={Notas para el examen 3},
 pdfkeywords={},
 pdfsubject={},
 pdfcreator={Emacs 26.3 (Org mode 9.1.9)}, 
 pdflang={Spanish}}
\begin{document}



\newcommand{\docente}{Arturo Arreola Alvarez}
\newcommand{\asignatura}{Organización de Computadoras (331)}
\newcommand{\semestre}{2022-2}

\newcommand{\miportada}[1]{
	\begin{titlepage}
		\vspace*{0.75in}
		\begin{flushleft}
			\sffamily
			\large #1       \\
			\Huge 
            \@title         \\
			\hrulefill
			\vspace{0.25in} \\
			\Large \@author \\
			\vspace*{\fill}
            \includegraphics[width=\textwidth]{../includes/filler.png} \\
			\vspace*{\fill}
			\large
			\begin{tabular}{|l|l|}
              \hline
			  Asignatura & \asignatura \\
			  Docente    & \docente    \\
			  Fecha      & \@date      \\
              \hline
			\end{tabular}
		\end{flushleft}
	\end{titlepage}
}

\miportada{ examen 3 }

\fancyhf{}
\lhead{ \asignatura }
\rhead{ \semestre }
\rfoot{Página \thepage}

\setlength\parindent{0pt}   % eliminar el intentado
\setlength{\parskip}{1.2em}
\maketitle

\section*{Pregunta 1}
\label{sec:orgca450a0}
En la seccion \texttt{.data} con la sintaxis:

\begin{verbatim}
gets_message  db   'Prueba de gets: ', 0xA, 0x0
;   ^         ^     \________________________/              
; nombre    tamaño              |
;                          Definicion   
\end{verbatim}

\begin{center}
\begin{tabular}{|l|l|l|}
\hline
Símbolo & Significado & Tamaño\\
\hline
\texttt{db} & define byte & 1 byte\\
\texttt{dw} & define word & 2 bytes\\
\texttt{dd} & define double word & 4 bytes\\
\texttt{dq} & define quad word & 8 bytes\\
\hline
\end{tabular}
\end{center}

\section*{Pregunta 2}
\label{sec:org59466aa}
En la seccion \texttt{.bss} con la sintaxis:
\begin{verbatim}
string_captura_getAlpha resb     255
;   ^                    ^        ^
; Nombre               Tamaño  Espacio reservado
\end{verbatim}

\begin{center}
\begin{tabular}{|l|l|l|}
\hline
Simbolo & Significado & Tamaño\\
\hline
\texttt{resb} & reserva bytes & 1 byte\\
\texttt{resw} & reserva words & 2 byte\\
\texttt{resd} & reserva double words & 4 byte\\
\texttt{redq} & reserva quad words & 8 byte\\
\hline
\end{tabular}
\end{center}

\section*{Pregunta 3}
\label{sec:org00f232a}
\begin{verbatim}
%macro delay 1                  ;subrutina con loop
    mov cx, %1
    nop
    nop
    nop
    nop
    nop	
%endmacro
\end{verbatim}
\end{document}
