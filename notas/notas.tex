% Created 2022-08-25 jue 11:15
% Intended LaTeX compiler: pdflatex
\documentclass[12pt]{article}
\usepackage[utf8]{inputenc}
\usepackage[T1]{fontenc}
\usepackage{graphicx}
\usepackage{grffile}
\usepackage{longtable}
\usepackage{wrapfig}
\usepackage{rotating}
\usepackage[normalem]{ulem}
\usepackage{amsmath}
\usepackage{textcomp}
\usepackage{amssymb}
\usepackage{capt-of}
\usepackage{hyperref}
\usepackage[spanish]{babel}
\usepackage{graphicx,geometry}
\geometry{ a4paper, left=1in, right=1in, top=1in, bottom=1in }
\renewcommand\familydefault{\sfdefault}
\usepackage{sectsty}
\sectionfont{\normalfont\Large }
\subsectionfont{\normalfont}
\usepackage{tabularx}
\usepackage{listings}
\lstdefinestyle{mystyle}{
numbers=left,
showspaces=false,
frame=leftline,
showspaces=false,
showstringspaces=false,
showtabs=false,
numberstyle=\tiny,
}
\lstset{
style=mystyle,
literate={á}{{\'a}}1
{é}{{\'e}}1
{í}{{\'{\i}}}1
{ó}{{\'o}}1
{ú}{{\'u}}1
{Á}{{\'A}}1
{É}{{\'E}}1
{Í}{{\'I}}1
{Ó}{{\'O}}1
{Ú}{{\'U}}1
{ü}{{\"u}}1
{Ü}{{\"U}}1
{ñ}{{\~n}}1
{Ñ}{{\~N}}1
{¿}{{?``}}1
{¡}{{!``}}1
}
\makeatletter
\usepackage{fancyhdr}
\pagestyle{fancy}
\usepackage{mdframed}
\BeforeBeginEnvironment{minted}{\begin{mdframed}}
\AfterEndEnvironment{minted}{\end{mdframed}}
\author{Luis Eduardo Galindo Amaya (1274895)}
\date{2022-08-25}
\title{Arquitectura De Computadoras}
\hypersetup{
 pdfauthor={Luis Eduardo Galindo Amaya (1274895)},
 pdftitle={Arquitectura De Computadoras},
 pdfkeywords={},
 pdfsubject={},
 pdfcreator={Emacs 26.3 (Org mode 9.1.9)}, 
 pdflang={Spanish}}
\begin{document}



\newcommand{\docente}{Arturo Arreola Alvarez}
\newcommand{\asignatura}{Organización de Computadoras (331)}
\newcommand{\semestre}{2022-2}

\newcommand{\miportada}[1]{
	\begin{titlepage}
		\vspace*{0.75in}
		\begin{flushleft}
			\sffamily
			\large #1       \\
			\Huge 
            \@title         \\
			\hrulefill
			\vspace{0.25in} \\
			\Large \@author \\
			\vspace*{\fill}
            \includegraphics[width=\textwidth]{../includes/filler.png} \\
			\vspace*{\fill}
			\large
			\begin{tabular}{|l|l|}
              \hline
			  Asignatura & \asignatura \\
			  Docente    & \docente    \\
			  Fecha      & \@date      \\
              \hline
			\end{tabular}
		\end{flushleft}
	\end{titlepage}
}

\miportada{ Notas }

\fancyhf{}
\lhead{ \asignatura }
\rhead{ \semestre }
\rfoot{Página \thepage}

\setlength\parindent{0pt}   % eliminar el intentado
\setlength{\parskip}{1.2em}
\maketitle

\section*{Bus de direcciones}
\label{sec:org6b91c20}
El bus de direcciones son las lineas de cobre dentro del procesador. Dependiendo de la cantidad de lineas que tenga podremos saber el tamaño de las dirección de memoria máxima direccionable 

\begin{verbatim}
          +-----------------+
          |                 |
Bus 1 ----+                 |
          |                 |
Bus 2 ----+      Main       |
          |                 |
Bus 3 ----+                 |
          |                 | 
          +-----------------+
\end{verbatim}
\captionof{figure}{\(2^3\) direcciones máximas direccionables.}


\section*{Bus de datos}
\label{sec:org1d454c0}
Nos permite saber si el tamaño de los registros, el bus de datos es bidireccional, puede dar y recibir datos el bus de datos es trifásico\footnote{Que tiene tres estados.}:  Leyendo, escribiendo y esperando.


\section*{Bus de control}
\label{sec:orgf58e8e4}
De cuatro a diez lineas en paralelo, les dice a los periféricos y la memoria que hacer, el bus de control es la manera mediante la cual se comunica la CPU con los otros dispositivos:

\begin{mdframed}
\begin{enumerate}
\item El CPU trae la información de memoria, las instrucciones están codificadas por lo que ocupan vario bytes
\item Decodifica la instrucción.
\item Determina las operaciones.
\item Trae el operador
\begin{itemize}
\item El dato por  el bus de datos
\item y la instrucción por el bus de control
\item el resultado sale por el bus de datos
\end{itemize}
\end{enumerate}
\end{mdframed}

\section*{Arquitecturas CISC y RISC}
\label{sec:orgb6cca51}
\subsection*{Computadora de conjunto de instrucciones complejas (CISC)}
\label{sec:orgd83a5b4}
\begin{itemize}
\item un conjunto de operaciones mas complejas en el procesador
\item las operaciones requieren mas de un ciclo del procesador
\item las operaciones tienen tamaños variables en memoria
\end{itemize}

\subsection*{Computadora de conjunto de instrucciones reducidas (RISC)}
\label{sec:orgca7c172}
\begin{itemize}
\item Cada operación se realiza en un solo ciclo de reloj
\item Tamaño constante de instrucciones, usualmente un byte
\item las operaciones son mas sencillas\footnote{Se parece mas a las operaciones en marie.js}
\end{itemize}

\section*{Segmentación encausada}
\label{sec:org307d8fe}
Podemos interpretarlo como una linea de ensamble dentro del procesador en la que el procesador va pasando las operaciones de una a una para evitar detenerse. 

\section*{Arquitectura Harvard}
\label{sec:org6541e61}


\section*{Arquitectura Von Neuman}
\label{sec:orgdfa73b5}
\end{document}
