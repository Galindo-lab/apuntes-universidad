% Created 2023-10-16 lun 22:01
% Intended LaTeX compiler: pdflatex
\documentclass[11pt]{article}
\usepackage[utf8]{inputenc}
\usepackage[T1]{fontenc}
\usepackage{graphicx}
\usepackage{grffile}
\usepackage{longtable}
\usepackage{wrapfig}
\usepackage{rotating}
\usepackage[normalem]{ulem}
\usepackage{amsmath}
\usepackage{textcomp}
\usepackage{amssymb}
\usepackage{capt-of}
\usepackage{hyperref}
\usepackage{../modern}
\bibliography{fuentes.bib}
\raggedbottom
\setcounter{secnumdepth}{2}
\author{Luis Eduardo Galidno Amaya}
\date{25 de Agosto del 2023}
\title{Meta 3.3. Conocer los sistemas criptográficos y sus diferentes características}
\hypersetup{
 pdfauthor={Luis Eduardo Galidno Amaya},
 pdftitle={Meta 3.3. Conocer los sistemas criptográficos y sus diferentes características},
 pdfkeywords={},
 pdfsubject={},
 pdfcreator={Emacs 27.1 (Org mode 9.3)}, 
 pdflang={Spanish}}
\begin{document}

\modentitlepage{../images/escudo-uabc-2022-1-tinta-pos.png}
\datasection{Individual}

\pagebreak


\section{Conclusión}
\label{sec:orgc6b426e}
Durante esta practica aprendí como identificar los diferentes sistemas de 
encriptacion desde los clásicos a los modernos y como los sistemas modernos 
se aprovechan de las matemáticas para encriptar los datos de manera mas segura
\end{document}
