% Created 2023-11-11 sáb 15:29
% Intended LaTeX compiler: pdflatex
\documentclass[11pt]{article}
\usepackage[utf8]{inputenc}
\usepackage[T1]{fontenc}
\usepackage{graphicx}
\usepackage{longtable}
\usepackage{wrapfig}
\usepackage{rotating}
\usepackage[normalem]{ulem}
\usepackage{amsmath}
\usepackage{amssymb}
\usepackage{capt-of}
\usepackage{hyperref}
\usepackage{../modern}
\bibliography{./sources.bib}
\raggedbottom
\setcounter{secnumdepth}{2}
\author{Luis Eduardo Galindo Amaya}
\date{sábado, 11 noviembre 2023}
\title{Meta 4.2 Instrucciones\\\medskip
\large Conocer que son los servicios de redes}
\hypersetup{
 pdfauthor={Luis Eduardo Galindo Amaya},
 pdftitle={Meta 4.2 Instrucciones},
 pdfkeywords={},
 pdfsubject={},
 pdfcreator={Emacs 28.1 (Org mode 9.5.2)}, 
 pdflang={Spanish}}
\begin{document}

\modentitlepage{../images/escudo-uabc-2022-1-tinta-pos.png}
\tableofcontents
\pagebreak
\datasection{Individual}

\section{¿Qué es un servicio de red?}
\label{sec:org45a1d55}
\autocite{Wikipedia_2023} Un servicio de red es la creación de
una red de trabajo en un ordenador. Generalmente los servicios de red
son instalados en uno o más firewalls del servidor seleccionado. Eso
facilita el uso y el fallo de muchos usuarios.

Un administrador de red debe mantener y desarrollar la infraestructura
de red de un negocio. Este se crea cuando se conecta dos o más equipos
de una red a través de cables a un eje central, o a través de
dispositivos inalámbricos para compartir información y recursos.

\section{Protocolo de Configuración Dinámica de Host (DHCP)}
\label{sec:org353cdb6}
Es un estándar utilizado para la configuración automática de
direcciones IP y otros parámetros en redes TCP/IP, reduciendo la carga
administrativa y la complejidad en la configuración de hosts. Permite
la gestión centralizada de direcciones IP y la comunicación entre
equipos clientes y servidores DHCP a través de agentes de
retransmisión. 

\section{Protocolo Simple de Administración de Red (SNMP)}
\label{sec:org814df06}
SNMP es el estándar utilizado para la gestión de redes
TCP/IP. Facilita la monitorización de la red a través de estaciones de
trabajo (gestores) que recopilan información de agentes pasivos
ubicados en dispositivos de red como routers y servidores. SNMP se
basa en el intercambio de mensajes conocidos como Protocolos de Unidad
de Datos (PDUs) y es ampliamente utilizado debido a su simplicidad. 

\section{Correo electrónico}
\label{sec:orgaae7237}
Los sistemas de correo electrónico constan de dos subsistemas: agentes
de usuario (MUA) y agentes de transferencia de mensajes (MTA). El MUA
es un programa que permite componer, recibir y gestionar correos
electrónicos, mientras que el MTA se encarga de transferir mensajes
entre hosts utilizando el Protocolo para la Transferencia Simple de
Correo Electrónico (SMTP). Los mensajes pueden pasar por varios MTAs
antes de llegar a su destino. 

\section{Domain Name System (DNS)}
\label{sec:org784a6b9}
El DNS es un servicio de Internet que traduce nombres de dominio y
URLs en direcciones IP. Esto facilita a los usuarios recordar sitios
web utilizando nombres en lugar de direcciones numéricas. El servidor
DNS procesa las solicitudes y busca en una base de datos para realizar
la conversión de nombres en direcciones IP. 

\section{Protocolo de Transferencia de Archivos (FTP)}
\label{sec:org2879929}
FTP es un protocolo que permite la transferencia de archivos entre
ordenadores a través de Internet. Utiliza TCP/IP y facilita la carga y
descarga de archivos. En una conexión FTP, se abre un puerto de
control en el servidor (puerto 21) y una segunda conexión de datos
desde el puerto 20 al cliente. FTP es útil para compartir archivos de
gran tamaño y utiliza el protocolo TCP para gestionar la transferencia
de datos. 

\section{Conclusión}
\label{sec:org3dff412}
A lo largo de la practica aprendi que son los servicios de red y para
que sirve y como cada uno brinda funcionalidades a las personas en la red.

\section{Referencias}
\label{sec:org4433765}
\printbibliography[heading=none]
\end{document}