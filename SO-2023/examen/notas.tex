% Created 2023-11-04 sáb 20:03
% Intended LaTeX compiler: pdflatex
\documentclass[11pt]{article}
\usepackage[utf8]{inputenc}
\usepackage[T1]{fontenc}
\usepackage{graphicx}
\usepackage{longtable}
\usepackage{wrapfig}
\usepackage{rotating}
\usepackage[normalem]{ulem}
\usepackage{amsmath}
\usepackage{amssymb}
\usepackage{capt-of}
\usepackage{hyperref}
\usepackage[top=1in,bottom=1in,papersize={8.5in,11in}]{geometry}
\usepackage[spanish, mexico]{babel}
\usepackage[backend=biber, style=apa]{biblatex}
\bibliography{./fuentes.bib}
\usepackage{listings}
\lstdefinestyle{mystyle}{
basicstyle=\ttfamily,
numbers=left,
showspaces=false,
frame=single,
showspaces=false,
showstringspaces=false,
showtabs=false,
numberstyle=\tiny,
aboveskip=25pt
%% aboveskip=\parskip
}
\lstset{
style=mystyle,
literate={á}{{\'a}}1
{é}{{\'e}}1
{í}{{\'{\i}}}1
{ó}{{\'o}}1
{ú}{{\'u}}1
{Á}{{\'A}}1
{É}{{\'E}}1
{Í}{{\'I}}1
{Ó}{{\'O}}1
{Ú}{{\'U}}1
{ü}{{\"u}}1
{Ü}{{\"U}}1
{ñ}{{\~n}}1
{Ñ}{{\~N}}1
{¿}{{?``}}1
{¡}{{!``}}1
}
\documentclass[14pt]{article}
\usepackage{setspace}
\usepackage[scaled]{helvet}
\usepackage[skip=10pt plus1pt, indent=40pt]{parskip}
\doublespacing
\setlength{\headheight}{12.5pt}
\renewcommand\familydefault{\sfdefault}
\setcounter{secnumdepth}{2}
\author{Luis Eduardo Galindo Amaya}
\date{jueves, 02 noviembre 2023}
\title{Notas\\\medskip
\large Administracion de sistemas operativos}
\hypersetup{
 pdfauthor={Luis Eduardo Galindo Amaya},
 pdftitle={Notas},
 pdfkeywords={},
 pdfsubject={},
 pdfcreator={Emacs 28.1 (Org mode 9.5.2)}, 
 pdflang={Spanish}}
\begin{document}

\maketitle
\setcounter{tocdepth}{2}
\tableofcontents


\section{Redireccionamientos}
\label{sec:orgfb8f067}
\subsection{Redireccionar salida sin concadenar (>)}
\label{sec:org17c35cf}
Redirecciona la salida a un archivo o dispositivo \texttt{>}, creando el archivo
si no existe y sobreescribiéndolo si ya existe.

\subsection{Redireccionar salida concadenando (>>)}
\label{sec:org18da9ea}
Redirecciona la salida estándar a un archivo o dispositivo \texttt{>>}, añadiendo
la salida al final del archivo.

\subsection{Redireccionar entrada (<)}
\label{sec:org0889e7f}
El simbolo \texttt{<} (entrada) Redirecciona stdin desde un archivo. El
contenido de un archivo es la entrada o input del comando. 

\subsection{Pasar salida como entrada Piping (|)}
\label{sec:orga7e941d}
El símbolo \texttt{|} (pipe) es un tipo de redireccionamiento ya que la salida
(stdout) de un comando es la entrada (stdin) de otro. Ejemplo:

\lstset{language=shell,label= ,caption= ,captionpos=b,numbers=none}
\begin{lstlisting}
grep -v "ORDINARIO" lista2023 | cut -d ':' -f3 
\end{lstlisting}

\begin{description}
\item[{grep -v ’ORDINARIO’ lista2023}] Se busca todos los elementos que \textbf{NO}
tengan ’ORDINARIO’

\item[{cut -d’:’ -f3}] Se obtiene la tercera columna
\end{description}

\subsection{Pasar salida a archivo y terminal (tee)}
\label{sec:org5982569}
El comando tee redirecciona la salida (stdout) a ambos, un archivo y a
la terminal. Se puede usar para supervisar la salida del comando

\subsection{Repaso de la practica}
\label{sec:org297c265}
\begin{description}
\item[{cat -n archivo.txt}] Manda el contenido de un archivo a la terminal
con las lineas numeradas (-n)

\item[{grep "substring" archivo.txt}] Busca el 'substring' en un flujo de
entrada, en este caso el archivo.txt

\item[{grep "RCA Records" billboard > RCA}] Busca el substring 'RCA
Records' en el archivo billboard y redirige la salida al archivo
'RCA'.

\item[{grep "Warner Records" billboard | tee Warner}] Busca el substring
'Warner Records' en el archivo billboard, guarda la salida en un
archivo llamado Warner y al mismo tiempo lo imprime en la terminal.
\end{description}

\section{{\bfseries\sffamily TODO} Repaso de commando de la unidad}
\label{sec:orgbce3516}
\subsection{Ordenar (sort)}
\label{sec:org0ba603b}
\begin{description}
\item[{sort -f <archivo>}] Ordena considerando de igual valor mayúsculas y minúsculas.

\item[{sort -M <archivo>}] Compara considerando los tres primeros caracteres de la línea como
el nombre de un mes en inglés.

\item[{sort -n <archivo>}] Ordena en forma numérica ascendente.

\item[{sort +1 <archivo>}] Ordena por la segunda columna. (+2 por la tercera, etc), considera
como delimitador el espacio y el tabulador.

\item[{sort –r <archivo>}] Invierte el orden.
\end{description}

\subsection{Buscar (grep)}
\label{sec:org3f134ba}
\begin{description}
\item[{grep <cadena> <archivo>}] Muestra la línea(s) donde encuentra la cadena.

\item[{grep -n <cadena> <archivo>}] Muestra la línea y el número de línea en donde encuentra la cadena.

\item[{grep -c <cadena> <archivo>}] Muestra cuántas líneas contienen el patrón especificado.

\item[{grep -v <cadena> <archivo>}] Muestra las líneas que no cumplen con el patrón de búsqueda.

\item[{grep -w <cadena> <archivo>}] Muestra las líneas que contienen la cadena como palabra completa.

\item[{grep –w <'frase'> <archivo>}] Muestra la línea donde se encuentra la frase completa.

\item[{grep -i <cadena><archivo>}] Evita la distinción entre mayúsculas y minúsculas.
\end{description}

\subsection{{\bfseries\sffamily TODO} Reemplazar (tr)}
\label{sec:orgb11d119}
\subsection{{\bfseries\sffamily TODO} Final (tail)}
\label{sec:org326cf52}
\subsection{{\bfseries\sffamily TODO} Inicio (cabeza)}
\label{sec:org9c5bf16}


\section{{\bfseries\sffamily TODO} Variables de Ambiente}
\label{sec:orgf1f57ca}
\subsection{Comandos para variables}
\label{sec:orge1f071d}
\begin{description}
\item[{crear una variable}] \texttt{variable=valor}

\item[{asignacion de valores}] \texttt{Variable=nuevovalor}

\item[{Exportación de variables del shell al ambiente}] \texttt{export Variable}

\item[{crear variable de ambiente con valor}] \texttt{export variable=valor}
\end{description}

\section{{\bfseries\sffamily TODO} Procesos}
\label{sec:org8cb6c4b}
Proceso es un concepto fundamental para todo sistema operativo. Es una
entidad dinámica que consiste en un programa en ejecución, sus valores
actuales, su estado y los recursos utilizados para manejar su
ejecución (memoria, CPU, dispositivos de E/S, etc). Pueden coexistir
varias instancias de un mismo programa en ejecución en forma
simultánea. ya que cada una de ellas es un proceso diferente. 

\pagebreak
\section{Hola mundo}
\label{sec:org321833c}
Aliquam erat volutpat.  Nunc eleifend leo vitae magna.  In id erat non
orci commodo lobortis.  Proin neque massa, cursus ut, gravida ut,
lobortis eget, lacus.  Sed diam.  Praesent fermentum tempor tellus.
Nullam tempus.  Mauris ac felis vel velit tristique imperdiet.  Donec
at pede.  Etiam vel neque nec dui dignissim bibendum.  Vivamus id
enim.  Phasellus neque orci, porta a, aliquet quis, semper a, massa.
Phasellus purus.  Pellentesque tristique imperdiet tortor.  Nam
euismod tellus id erat. 

Nullam eu ante vel est convallis dignissim.  Fusce suscipit, wisi nec
facilisis facilisis, est dui fermentum leo, quis tempor ligula erat
quis odio.  Nunc porta vulputate tellus.  Nunc rutrum turpis sed pede.
Sed bibendum.  Aliquam posuere.  Nunc aliquet, augue nec adipiscing
interdum, lacus tellus malesuada massa, quis varius mi purus non odio.
Pellentesque condimentum, magna ut suscipit hendrerit, ipsum augue
ornare nulla, non luctus diam neque sit amet urna.  Curabitur
vulputate vestibulum lorem.  Fusce sagittis, libero non molestie
mollis, magna orci ultrices dolor, at vulputate neque nulla lacinia
eros.  Sed id ligula quis est convallis tempor.  Curabitur lacinia
pulvinar nibh.  Nam a sapien. 

Aliquam erat volutpat.  Nunc eleifend leo vitae magna.  In id erat non
orci commodo lobortis.  Proin neque massa, cursus ut, gravida ut,
lobortis eget, lacus.  Sed diam.  Praesent fermentum tempor tellus.
Nullam tempus.  Mauris ac felis vel velit tristique imperdiet.  Donec
at pede.  Etiam vel neque nec dui dignissim bibendum.  Vivamus id
enim.  Phasellus neque orci, porta a, aliquet quis, semper a, massa.
Phasellus purus.  Pellentesque tristique imperdiet tortor.  Nam
euismod tellus id erat. 


Nullam eu ante vel est convallis dignissim.  Fusce suscipit, wisi nec
facilisis facilisis, est dui fermentum leo, quis tempor ligula erat
quis odio.  Nunc porta vulputate tellus.  Nunc rutrum turpis sed pede.
Sed bibendum.  Aliquam posuere.  Nunc aliquet, augue nec adipiscing
interdum, lacus tellus malesuada massa, quis varius mi purus non odio.
Pellentesque condimentum, magna ut suscipit hendrerit, ipsum augue
ornare nulla, non luctus diam neque sit amet urna.  Curabitur
vulputate vestibulum lorem.  Fusce sagittis, libero non molestie
mollis, magna orci ultrices dolor, at vulputate neque nulla lacinia
eros.  Sed id ligula quis est convallis tempor.  Curabitur lacinia
pulvinar nibh.  Nam a sapien. 
\end{document}