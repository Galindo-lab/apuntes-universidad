% Created 2023-11-09 jue 22:34
% Intended LaTeX compiler: pdflatex
\documentclass[11pt]{article}
\usepackage[utf8]{inputenc}
\usepackage[T1]{fontenc}
\usepackage{graphicx}
\usepackage{longtable}
\usepackage{wrapfig}
\usepackage{rotating}
\usepackage[normalem]{ulem}
\usepackage{amsmath}
\usepackage{amssymb}
\usepackage{capt-of}
\usepackage{hyperref}
\usepackage{../modern}
\bibliography{./fuentea.bib}
\raggedbottom
\setcounter{secnumdepth}{2}
\author{Luis Eduardo Galindo Amaya}
\date{jueves, 09 noviembre 2023}
\title{Taller 7. DNS}
\hypersetup{
 pdfauthor={Luis Eduardo Galindo Amaya},
 pdftitle={Taller 7. DNS},
 pdfkeywords={},
 pdfsubject={},
 pdfcreator={Emacs 28.1 (Org mode 9.5.2)}, 
 pdflang={Spanish}}
\begin{document}

\modentitlepage{../images/escudo-uabc-2022-1-tinta-pos.png}
\datasection{Individual}

\section{Cuestionario}
\label{sec:org4910d09}
\begin{description}
\item[{¿Qué es un nombre de dominio?}] Un nombre de dominio es una cadena
de texto que se asigna a una dirección IP numérica, que se utiliza
para acceder a un sitio web.

\item[{¿Cuáles son las reglas para elegir un nombre de dominio?}] \textasciitilde{}

\item Los únicos caracteres permitidos para un nombre de dominio son:
\begin{itemize}
\item Los pertenecientes al alfabeto inglés: de la a a la z.
\item Los dígitos del 0 al 9 (No es aconsejable un nombre con sólo dígitos).
\item El guión: - (No puede ser ni el primero ni el último caràcter del nombre).
\item No se admiten: acentos, diéresis, la ñ, la ç, espacios en
blanco, el punto, etc. No hay distinción entre minúsculas y mayúsculas.
\end{itemize}

\item Las longitudes màximas y mínimas de un nombre de dominio son:
\begin{itemize}
\item Para los gTLD .com, .org, .net, hay un máximo de 64 y un mínimo de 2.
\item Para el TLD ISO-3166 .es: como máximo 63 y como mínimo 3.
\item Para los restantes TLD ISO-3166 depende del registro.
\end{itemize}

\item[{¿Cómo se registra un nombre de dominio?}] \textasciitilde{}
\begin{itemize}
\item Los nombres de dominio pueden registrarse a través de diversas
empresas conocidas como "registradores" que compiten entre sí.

\item Puedes encontrar una lista de estas empresas en el Directorio de
Registradores en este sitio.

\item El registrador que elijas te solicitará proporcionar datos de
contacto y técnicos para completar el registro.

\item El registrador mantendrá registros de la información de contacto y
enviará la información técnica a un directorio central llamado
"registro".

\item El registro facilita a otras computadoras en Internet la
información necesaria para enviarte correos electrónicos o
encontrar tu sitio web.

\item Además, se te requerirá firmar un contrato de registro con el
registrador que establecerá los términos bajo los cuales se acepta
y mantendrá tu registro.
\end{itemize}
\end{description}


\begin{description}
\item[{¿Cuál es el costo de un dominio web?}] entre 10 y 15 dólares al
año aproximadamente.

\item[{¿Cómo funciona un DNS?}] Los servidores DNS convierten las
solicitudes de nombres en direcciones IP, con lo que se controla a
qué servidor se dirigirá un usuario final cuando escriba un nombre
de dominio en su navegador web. Estas solicitudes se denominan
consultas.

\item[{¿Qué es un DNS resolver o solucionador de DNS?}] Un DNS resolver es
un servicio que proporciona una dirección IP cuando se solicita un
nombre de dominio. Se habla de resolver un dominio obteniendo su
dirección IP; “resolver” en inglés se define en el documento de
especificación de Internet RFC 1034.

\item[{¿Qué es un DNS root name server o servidor de nombres de raíz de DNS?}] Un servidor raíz es un servidor de nombres para la zona raíz del
Sistema de nombres de dominio de Internet.

\item[{¿Qué es un TLD o top-level domain?}] Un dominio del nivel superior
o TLD es la más alta categoría de los FQDN que es traducida a
direcciones IP por los DNS oficiales de Internet. Los nombres
servidos por los DNS oficiales son administrados por la Internet
Corporation for Assigned Names and Numbers.

\item[{¿Qué es la ICANN?}] ICANN es una organización que opera a nivel
multinacional/internacional y es la responsable de asignar las
direcciones del protocolo IP, de los identificadores de protocolo,
de las funciones de gestión del sistema de dominio y de la
administración del sistema de servidores raíz.

\item[{¿Cuáles son los ataques más comunes a un DNS?}] \textasciitilde{}
\begin{itemize}
\item Suplantación de DNS/envenenamiento de caché.
\item Túnel de DNS.
\item Secuestro de DNS.
\item Ataque NXDOMAIN.
\item Ataque de dominio fantasma.
\item Ataque de subdominio aleatorio.
\item Ataque de bloqueo de dominio.
\item Ataque CPE basado en red de robots (botnet).
\end{itemize}
\end{description}

\section{Conclusión}
\label{sec:org939a19d}
Durante esta practica aprendí que es un nombre de dominio y para que
se utiliza, que es la ICANN y en que se relaciona con el internet
ademas del costo de estos. Los dominios nos permiten nombrar de manera
mas facil nuestras direcciones 

\section{Fuentes}
\label{sec:orgd53522f}
\begin{itemize}
\item \url{https://www.icann.org/resources/pages/faqs-84-2012-02-25-en\#2}
\item \url{https://www.hostinger.mx/tutoriales/cuanto-cuesta-un-dominio-web}
\item \url{https://aws.amazon.com/es/route53/what-is-dns/\#:\~:text=Los\%20servidores\%20DNS\%20convierten\%20las,Estas\%20solicitudes\%20se\%20denominan\%20consultas}
\item \url{https://www.ionos.mx/digitalguide/servidores/know-how/dns-resolver/\#:\~:text=Los\%20DNS\%20resolvers\%20son\%20un,y\%20los\%20servidores\%20de\%20nombres}.
\item \url{https://es.wikipedia.org/wiki/Corporaci\%C3\%B3n\_de\_Internet\_para\_la\_Asignaci\%C3\%B3n\_de\_Nombres\_y\_N\%C3\%BAmeros}
\item \url{https://www.cloudflare.com/es-es/learning/dns/dns-security/\#:\~:text=Estas\%20limitaciones\%2C\%20combinadas\%20con\%20los,interceptaci\%C3\%B3n\%20de\%20informaci\%C3\%B3n\%20personal\%20privada}.
\end{itemize}
\end{document}