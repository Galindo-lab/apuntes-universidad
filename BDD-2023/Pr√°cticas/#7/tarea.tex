% Created 2023-10-31 mar 01:24
% Intended LaTeX compiler: pdflatex
\documentclass[11pt]{article}
\usepackage[utf8]{inputenc}
\usepackage[T1]{fontenc}
\usepackage{graphicx}
\usepackage{grffile}
\usepackage{longtable}
\usepackage{wrapfig}
\usepackage{rotating}
\usepackage[normalem]{ulem}
\usepackage{amsmath}
\usepackage{textcomp}
\usepackage{amssymb}
\usepackage{capt-of}
\usepackage{hyperref}
\usepackage[spanish, mexico]{babel}
\usepackage[top=1in,bottom=1in,papersize={8.5in,11in}]{geometry}
\usepackage[scaled]{helvet}
\renewcommand\familydefault{\sfdefault}
\setlength{\parindent}{0pt}
\usepackage{listings}
\lstdefinestyle{mystyle}{basicstyle=\ttfamily,numbers=left,showspaces=false,frame=single,showspaces=false,showstringspaces=false,showtabs=false,numberstyle=\tiny,aboveskip=25pt}
\let\verbatim\someundefinedcommand\lstnewenvironment{verbatim}{}{}
\setcounter{secnumdepth}{2}
\author{Luis Eduardo Galindo Amaya (1274895)}
\date{viernes, 27 octubre 2023}
\title{Práctica 7\\\medskip
\large Funciones, Procedimientos Almacenados y Triggers}
\hypersetup{
 pdfauthor={Luis Eduardo Galindo Amaya (1274895)},
 pdftitle={Práctica 7},
 pdfkeywords={},
 pdfsubject={},
 pdfcreator={Emacs 27.1 (Org mode 9.3)}, 
 pdflang={Spanish}}
\begin{document}

\maketitle
\tableofcontents

\pagebreak

\section{Función para Calcular Años Transcurridos}
\label{sec:org0d84073}
Crear un FUNCIÓN que reciba una fecha y con base a ella, calcule los
años transcurridos.

\begin{verbatim}
DELIMITER $ 
CREATE FUNCTION fn_anios_transcurridos (fecha_inicial date)
RETURNS int deterministic
BEGIN
    DECLARE days_diff int;
    DECLARE anios_diff int;
    SET days_diff = datediff(now(), fecha_inicial);

    IF (
        month(fecha_inicial) > month(now())
        OR (
            month(fecha_inicial) = month(now()) 
            AND day(fecha_inicial) > day(now())
        )
       )
    THEN
      SET anios_diff = days_diff / 365 - 1;
    ELSE
       SET anios_diff = days_diff / 365;
    END IF;

    RETURN anios_diff;
END
$
\end{verbatim}

\section{Procedimiento Almacenado para Calcular Edad y Años de Antigüedad}
\label{sec:org1cde22f}
Crear un PROCEDIMIENTO ALMACENADO en el cual con base a la tabla employees, y 
los campos de birth\_date y hire\_date calcule la edad y los años de antigüedad 
respectivamente, utilizando la función que creaste.

\begin{verbatim}
DELIMITER $ 
CREATE PROCEDURE sp_edad_y_anios_trabajados ()
BEGIN
    SELECT
        emp_no AS Numero_Empleado,
        concat(first_name, ",", last_name) AS Nombre_Empleado,
        gender AS Genero,
        fn_anios_transcurridos (birth_date) AS Edad,
        fn_anios_transcurridos (hire_date) AS Antiguedad,
        (CASE
            WHEN fn_anios_transcurridos (birth_date) BETWEEN 60 AND 64 THEN
                "Puede iniciar su proceso de pensión"
            WHEN fn_anios_transcurridos (birth_date) > 64 THEN
                "Debe iniciar su proceso de pensión"
            ELSE
                -- Antiguedad < 60
                "No es candidato para pensionarse"
            END) AS Observacion
    FROM
        employees;

END 
$
\end{verbatim}

\section{Procedimiento Almacenado para Estadísticas de Edad y Pensión}
\label{sec:org7fbf290}
Crear un PROCEDIMIENTO ALMACENADO que muestre la estadística con base a la edad
y los criterios establecidos en el requerimiento 2, nos diga:

\begin{itemize}
\item ¿Cuántos hombres y Mujeres NO son candidatos al proceso de pensión?
\item ¿Cuántos hombres y Mujeres PUEDEN iniciar su proceso de pensión?
\item ¿Cuántos hombres y Mujeres DEBEN iniciar su proceso de pensión?
\end{itemize}

\subsection{Procedimiento}
\label{sec:orga28cc52}
\begin{verbatim}
DELIMITER $
CREATE PROCEDURE sp_estadistica_pension ()
BEGIN
    DECLARE per_fac int;
    -- factor que convierte una cantidad de empleados a porcentaje
    SET per_fac = 100 / (SELECT count(*) FROM employees);

    -- Calcular el número de empleados
    CREATE TEMPORARY TABLE IF NOT EXISTS Freq_Pension AS
    SELECT
        -- Nombre completo del genero
        (
            CASE WHEN gender = "M" THEN
                "MASCULINO"
            WHEN gender = "F" THEN
                "FEMENINO"
            ELSE
                "INDEFINIDO"
            END) AS Genero,


        sum(fn_anios_transcurridos (birth_date) < 60)
                                   AS NO_Son_Candidatos,

        sum(fn_anios_transcurridos (birth_date) BETWEEN 60 AND 65)
                                   AS PUEDEN_Iniciar_Proceso_Pension,

        sum(fn_anios_transcurridos (birth_date) > 65)
                                   AS DEBEN_Iniciar_Proceso_Pension
    FROM
        employees
    GROUP BY
        gender;

    -- Calcular porcentajes
    SELECT
        Genero,
        NO_Son_Candidatos,
        (NO_Son_Candidatos * per_fac)
                                   AS Per_NO_Son_Candidatos,

        PUEDEN_Iniciar_Proceso_Pension,
        (PUEDEN_Iniciar_Proceso_Pension * per_fac)
                                   AS PER_PUEDEN_Iniciar_Proceso_Pension,

        DEBEN_Iniciar_Proceso_Pension,
        (DEBEN_Iniciar_Proceso_Pension * per_fac)
                                   AS PER_DEBEN_Iniciar_Proceso_Pension
    FROM
        Freq_Pension;

    -- Eliminar la tabla temporal
    DROP TABLE Freq_Pension;
END
$
\end{verbatim}

\subsection{Cuestionario}
\label{sec:orgc88288b}
\subsubsection*{¿Cuántos hombres y mujeres NO son candidatos al proceso de pensión?}
\label{sec:orge16c82e}
\begin{itemize}
\item 1 hombre  y 1 mujeres
\end{itemize}

\subsubsection*{¿Cuántos hombres y mujeres PUEDEN iniciar su proceso de pensión?}
\label{sec:org12df5d4}
\begin{itemize}
\item 18 hombres y 9 mujeres
\end{itemize}

\subsubsection*{¿Cuántos hombres y mujeres DEBEN iniciar su proceso de pensión?}
\label{sec:org5f5d5ce}
\begin{itemize}
\item 14 hombres y 7 mujeres
\end{itemize}

\section{Trigger BEFORE INSERT para asignar categoría}
\label{sec:org84c3e48}
Crear un TRIGGER con el tiempo y evento BEFORE INSERT donde antes de insertar 
un nuevo empleado a la tabla employee, se deberá asignar al campo Category el 
valor 1. 

El campo Category en la tabla employee no existe, por lo tanto, tendrás que 
agregarlo, las categorías irán de la 1 a la 8.

\subsection{Agregar el campo categoria}
\label{sec:org2cb831e}
El campo 'Category' en la tabla employee no existe, por lo tanto, tendrás que
agregarlo, las categorías irán de la 1 a la 8.

\begin{verbatim}
ALTER TABLE employees   
ADD category INT CHECK (category >= 1 AND category <= 8);
\end{verbatim}

\subsection{Trigger BEFORE INSERT}
\label{sec:org3a4bb22}
\begin{verbatim}
DELIMITER %
CREATE TRIGGER before_set_category
    BEFORE INSERT ON employees
    FOR EACH ROW
BEGIN
    SET new.category = 1;
END
%
\end{verbatim}

\subsection{Prueba del trigger}
\label{sec:org68a8fcd}
\begin{verbatim}
INSERT INTO employees
       (emp_no, birth_date, first_name, last_name, gender, hire_date)
VALUES (20000, "31-08-2001", "Luis", "Galindo", "M", now());
\end{verbatim}

\section{Trigger AFTER INSERT para Generar Registro en Salaries}
\label{sec:orga44192e}
Crear un TRIGGER con el tiempo y evento AFTER INSERT que después de INSERTAR un
registro en la tabla employees, genere un registro en la tabla salaries, 
agregando en from\_date la fecha actual, en to\_date dejarlo NULL y en salary 
establecer el valor 5000.

\subsection{Permitir NULL en el campo to\_date}
\label{sec:orgc7bb4de}
\begin{verbatim}
ALTER TABLE salaries MODIFY to_date DATE NULL;  
\end{verbatim}

\subsection{Trigger AFTER INSERT}
\label{sec:orgfb652e0}
\begin{verbatim}
DELIMITER %
CREATE TRIGGER after_add_salary
    AFTER INSERT ON employees
    FOR EACH ROW
BEGIN
    INSERT INTO salaries
    (emp_no, salary, from_date, to_date)
VALUES
    (new.emp_no, 5000, now(), null);
END
%
\end{verbatim}

\section{Trigger BEFORE UPDATE con Auditoría de Cambios en Categoría}
\label{sec:org7c77511}
Crear un TRIGGER con el tiempo y evento BEFORE UPDATE donde antes de ACTUALIZAR
un registro en la tabla employees, genere un registro en la tabla 
employee\_category\_audit, agregando para dicho empleado la categoría nueva, la 
categoría vieja y la fecha en la que se realizó dicho cambio. Esta tabla 
\textbf{NO EXISTE} así que tendrás que crearla.

\subsection{Crear tabla employee\_category\_audit}
\label{sec:orgfda5578}
\begin{verbatim}
CREATE TABLE if not exists employee_category_audit (
    emp_no int not null,
    change_date date,
    old_category int,
    new_Category int,

    CONSTRAINT FK_emp_no FOREIGN KEY (emp_no) 
    references employees(emp_no)
);  
\end{verbatim}

\subsection{Trigger BEFORE UPDATE}
\label{sec:org357f41d}
\begin{verbatim}
DELIMITER %
CREATE TRIGGER after_category_update
    BEFORE UPDATE ON employees
    FOR EACH ROW
BEGIN
    IF old.category <> new.category THEN
        INSERT INTO employee_category_audit
        (emp_no, change_date, old_category, new_Category)
    VALUES
        (new.emp_no, now(), old.category, new.category);
    END IF;
END
%
\end{verbatim}

\section{Trigger AFTER UPDATE para Incrementar Salario y Fecha}
\label{sec:org3101bac}
\subsection{Eliminar las llaves primarias de salaries}
\label{sec:org7d4f30d}
\begin{verbatim}
ALTER TABLE salaries DROP CONSTRAINT salaries_ibfk_1;
ALTER TABLE salaries DROP CONSTRAINT `PRIMARY`;
\end{verbatim}

\subsection{Trigger}
\label{sec:org5f81c8f}
\begin{verbatim}
DELIMITER %
CREATE TRIGGER after_set_category
    AFTER UPDATE ON employees
    FOR EACH ROW
BEGIN
    DECLARE old_salary INT;
    DECLARE salary_fac FLOAT;
    DECLARE aplication_date date;

    -- ultimo salario
    SET old_salary = (SELECT salary FROM salaries
        WHERE old.emp_no AND to_date IS NULL);

    -- factor de porcentaje
    SET salary_fac = (old_salary / 100);

    -- si es sabado o domingo
    SET aplication_date = (CASE
        WHEN DAYOFWEEK (now()) = 1 THEN
             date_add (now(), interval 1 day) -- domingo
        WHEN DAYOFWEEK (now()) = 7 THEN
             date_add (now(), interval 2 day) -- sabado
        ELSE
            now()
    END);

IF old.category <> new.category THEN
    -- Actualizar fecha del salario
    UPDATE salaries SET to_date = now()
    WHERE emp_no = old.emp_no AND to_date IS NULL;

    -- insertar el nuevo saliario
    INSERT INTO salaries (emp_no, from_date, to_date, salary)
           VALUES (new.emp_no, aplication_date, NULL, (
            -- calcular nuevo salario
            CASE WHEN new.category = 1 THEN
                old_salary + salary_fac * 5  -- + 5%
            WHEN new.category BETWEEN 2 AND 4 THEN
                old_salary + salary_fac * 10 -- +10%
            WHEN new.category BETWEEN 5 AND 7 THEN
                old_salary + salary_fac * 30 -- +30%
            WHEN new.category = 8 THEN
                old_salary + salary_fac * 60 -- +60%
            END));
    END IF;
END
%
\end{verbatim}

\subsection{Probar el trigger}
\label{sec:org4964b13}
\begin{verbatim}
-- crear al empleado
INSERT INTO employees
       (emp_no, birth_date,first_name,last_name,gender,hire_date,category)
VALUES 
       (20000, "31-08-2001", "Luis", "Galindo", "M", now(), 0);

-- Agregar un salario 
INSERT INTO salaries
       (emp_no,salary,from_date,to_date) 
VALUES 
       (20000, 1000, now(), NULL);

-- Actualizar datos
UPDATE employees SET category=3 WHERE emp_no=20000;
\end{verbatim}
\end{document}
