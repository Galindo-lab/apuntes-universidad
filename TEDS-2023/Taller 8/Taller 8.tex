% Created 2023-10-25 mié 22:01
% Intended LaTeX compiler: pdflatex
\documentclass[11pt]{article}
\usepackage[utf8]{inputenc}
\usepackage[T1]{fontenc}
\usepackage{graphicx}
\usepackage{grffile}
\usepackage{longtable}
\usepackage{wrapfig}
\usepackage{rotating}
\usepackage[normalem]{ulem}
\usepackage{amsmath}
\usepackage{textcomp}
\usepackage{amssymb}
\usepackage{capt-of}
\usepackage{hyperref}
\input{~/.emacs.d/templates/modern.sty}
\setcounter{secnumdepth}{2}
\author{Luis Eduardo Galindo Amaya (1274895)}
\date{miércoles, 25 octubre 2023}
\title{Taller 8. Descripción del problema}
\hypersetup{
 pdfauthor={Luis Eduardo Galindo Amaya (1274895)},
 pdftitle={Taller 8. Descripción del problema},
 pdfkeywords={},
 pdfsubject={},
 pdfcreator={Emacs 27.1 (Org mode 9.3)}, 
 pdflang={Spanish}}
\begin{document}

\makeatletter
\makeatletter
\modentitlepage{~/.emacs.d/templates/img/tinta.png}
\tableofcontents
\pagebreak
\datasection{Individual}

\section{Descripción del Problema}
\label{sec:orgc831262}
En la actualidad uno de los mayores desafíos que tienen los hospitales es la 
organización, gestionar la mayor cantidad de pacientes de la manera mas ordenada 
posible, además de capturar sus datos de forma sencilla. Integrar los problemas
que presenta la organización de los consultorios de un hospital permitiría sin
dudas agilizar los procesos internos reduciendo los tiempos entre consultas.

\section{Descripción de la Solución}
\label{sec:org20499e3}
La solución que se plantea consiste en integrar varias partes básicas para la 
organización de un hospital, integrarlos permitiría pasar información rápidamente 
entre los sistemas, supervisar las diferentes etapas de los procesos y  :

\subsection{Captura de pacientes}
\label{sec:org6debec6}
El sistema captura los datos de los pacientes al llegar a la clínica para 
agregarlos a la fila de espera.

\subsection{Priorizacion de pacientes}
\label{sec:org2b9784e}
Se les asignara una prioridad al llegar las, personas de mayor edad, embarazadas
y emergencias deben pasar primero.

\subsection{Gestión de consultorios}
\label{sec:orge1a9264}
al ser el turno del paciente se le asignara al consultorio disponible. 

\subsection{Captura de historiales médicos}
\label{sec:org413ff59}
El sistema captura los historiales médicos de una manera mas sistemática acorde
a el tipo de hospital que esta utilizando el sistema.

\subsection{Emisión de recetas médicas}
\label{sec:orgc3770c2}
Al terminar la consulta el sistema emite una receta al correo del paciente.

\subsection{Análisis de los datos}
\label{sec:orgde89aa1}
Una vez con el sistema en linea se puede mantener una supervisión constante de 
las clínicas y de los pacientes atendidos en ellas mediante un dashboard.


\pagebreak

\section{Tecnologías a aplicar}
\label{sec:org1a083be}
\subsection{Django y Python}
\label{sec:orgb9e9f1d}
Para controlar la base de datos y las vistas se usara django, soluciona 
principalmente todo lo relacionado al modelo MVC. Django esta hecho con python
por lo que el desarollo principalmente se hará en python. 

\begin{figure}[htbp]
\centering
\includegraphics[width=10cm]{img/python-django.png}
\caption{Django y Python}
\end{figure}

\subsection{Docker}
\label{sec:org24ee138}
Para el despliegue de la aplicación se usara Docker, docker es una tecnología 
de Contenderían de aplicaciones la cual nos hace no ser tan dependientes 
del hardware del usuario lo cual nos permite únicamente enfocarnos en el 
desarrollo. 

\subsection{Bootsrap}
\label{sec:org7da8b2f}
Bootsrap es un framework de css para web que permite crear interfaces web de 
manera sencilla, una de las ventajas de usar este framework es que no tenemos 

\subsection{MongoDB}
\label{sec:org5b0ef68}
MongoDB es un sistema de base de datos NoSQL, orientado a documentos y de código 
abierto. A comparación de otras bases de datos Mongo almacena sus registros como
archivos individuales

\section{Aplicaciones futuras}
\label{sec:org405f85d}
\subsection{Análisis de datos}
\label{sec:orgaea0120}
Una de las posibles aplicaciones secundarias de nuestra solución es analizar
los diagnósticos de los pacientes y sus recetas para poder predecir que cosas
se deben solicitar en cada época del años así como detectar epidemias. 

\subsection{Estimación de tiempos}
\label{sec:org7eaa545}
Utilizando da cantidad de doctores, los tiempos en los que trabajan el sistema 
podría hacer estimaciones del tiempo que tomara la consulta desde que se 
solicita.

\section{Tecnologías con las que se puede complementar}
\label{sec:orgc7ba471}
\subsection{Ciencia de datos}
\label{sec:org4730ad9}
Al se un sistema que recolecta datos se podría utilizar para capturar datos y 
crear datasets que pueden permitir hacer ciencia de datos, claramente 
respetando la ley de protección de datos y viendo por la privacidad del 
usuario en todo momento.

\subsection{Sistemas expertos}
\label{sec:org4fabd26}
La recopilación de datos puede ayudar a la creación de sistemas expertos que 
puedan asistir a los doctores de manera que puedan hacer preguntas mas 
precisa para poder hacer diagnósticos mas efectivos.

\section{Conclusión}
\label{sec:org2f39734}
Durante esta practica pude reflexionar como las tecnologías que creamos pueden
ayudar a las personas a poder hacer mas con las tecnología y como identificar 
los problemas que se pueden solucionar tecnologías que dominamos.  
\end{document}
