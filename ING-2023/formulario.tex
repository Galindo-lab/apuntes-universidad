% Created 2023-03-14 mar 15:24
% Intended LaTeX compiler: pdflatex
\documentclass[11pt]{article}
\usepackage[utf8]{inputenc}
\usepackage[T1]{fontenc}
\usepackage{graphicx}
\usepackage{grffile}
\usepackage{longtable}
\usepackage{wrapfig}
\usepackage{rotating}
\usepackage[normalem]{ulem}
\usepackage{amsmath}
\usepackage{textcomp}
\usepackage{amssymb}
\usepackage{capt-of}
\usepackage{hyperref}
\usepackage[margin=10mm,papersize={8.5in,11in}]{geometry}
\author{Luis Eduardo Galindo Amaya}
\date{2023-03-03}
\title{Formulario}
\hypersetup{
 pdfauthor={Luis Eduardo Galindo Amaya},
 pdftitle={Formulario},
 pdfkeywords={},
 pdfsubject={},
 pdfcreator={Emacs 27.1 (Org mode 9.3)}, 
 pdflang={English}}
\begin{document}


\section*{Formulario ingeniería económica}
\label{sec:org067f403}
\begin{center}
\begin{tabular}{lll}
\(P\) = Cantidad inicial & \(n\) = Numero de periodos & \(CB\) = cantidad Base\\
\(I\) = Interés & \(A\) = Anualidad & \(n\) = cantidad en el periodo n\\
\(G\) = Gradiente aritmetico &  & \\
\end{tabular}

\end{center}

\section*{Interes basico}
\label{sec:orgb678703}
\begin{itemize}
\item \(F-P\) = Interes (I)
\item \(I / P\) = Tasa de interes (i)
\item \(100 \cdot i\) = Tasa de interes porcentual (i\%)
\item \(F / (i+1)\) = Cantidad inicial (P)
\item \(P\cdot (i+ 1)\) = Cantidad final (F)
\end{itemize}

\section*{Interés simple}
\label{sec:org3a643ed}
\begin{itemize}
\item \(P\cdot(1+i\cdot n)\) = Futuro dado un presente (F)
\item \(F / (1+i\cdot n)\) = Presente dado un futuro (P)
\end{itemize}

\section*{Interés compuesto}
\label{sec:org631030c}
\begin{itemize}
\item \(P\cdot (1+i)^n\) = Futuro dado un presente (F/P)
\item \(F / (1+i)^n\) = Presente dado un futuro (P/F)
\end{itemize}

\section*{Anualidad}
\label{sec:org6c0f422}
\begin{itemize}
\item \(A~\frac{(1+i)^n-1}{i}\) = Futuro dado una anualidad (F/A)
\item \(A~\frac{(1+i)^n-1}{(1+i)^n\cdot i}\) = Presente dado una anualidad (P/A)
\item \(F~\frac{i}{(1+i)^n-1}\) = Anualidad dada un futuro (A/F)
\item \(P~\frac{(1+i)^n\cdot i}{(1+i)^n-1}\) = Anualidad dado un presente (A/P)
\end{itemize}

\section*{Gradiente aritmetico}
\label{sec:orgd0404ac}
\begin{itemize}
\item \(\frac{C_n - CB}{n-1}\) = Gradiente (G)
\item \(G\left[\frac{(1+i)^n-i\cdot n-1}{i^2(1+i)^n}\right]\) = Presente dado un gradiente (P\textsubscript{G}) (P/G)
\item \(G\left[\frac{1}{i}-\frac{n}{(1+i)^n-1}\right]\) = Anualidad dado un gradiente (A\textsubscript{G}) (A/G)
\item \(G\left[ \frac{1}{i} \left( \frac{(1+i)^n-1}{i} - n \right) \right]\) = Futuro dado un gradiente (F\textsubscript{G}) (F/G)
\end{itemize}

\section*{Convertir interés de un periodo a otro\footnote{\(i_a\) con un periodo grande, \(i\) con periodo pequeño, \(k\) es la frecuencia entre los periodos}}
\label{sec:org9b3a25f}
\begin{itemize}
\item \(i_a = (1+i)^k - 1\)
\item \(i=(1+i_a)^{\frac{1}{k}}-1\)
\end{itemize}
\end{document}
